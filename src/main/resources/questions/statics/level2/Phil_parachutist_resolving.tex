%% ID: A1986PIQ1l
%% TITLE: A parachutist
%% TYPE: question
%% QUESTIONTYPE:  numerical
%% CONCEPTS: forces, vectors1, newtoni
%% VIDEOS: 
%% LEVEL: 2
%% TOPIC: mechanics/statics
%% ORDER: 5

\begin{problem}[Phil_parachutist_resolving]
{\exposition{A parachutist of mass 60 kg is falling to Earth at a constant speed. She is supported by a parachute of mass 15 kg.} \question{ Find the magnitude of the resistive force acting upwards.}
}
{\textit{Created for the Rutherford School Physics Project by PS.}}
{The parachutist is falling at constant speed, so there is no acceleration; therefore, the forces acting on her must be balanced. The downwards force consists of the weight of the parachutist plus the weight of the parachute. The total mass of the parachustist and the parachute is 75 kg, so the total weight $= mg = 75 \times 9.8 = 735$ N. If the forces are balanced then the weight must be equal to the resistive force acting upwards. \answer{The magnitude of the resistive force is 735 N.}
}
\end{problem}