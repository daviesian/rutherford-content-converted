%% ID: A1984PIQ1l
%% TITLE: Speedometer on a Bike Wheel
%% TYPE: question
%% QUESTIONTYPE: numeric
%% CONCEPTS: angular_vel
%% LEVEL: 4
%% TOPIC: mechanics/circular
\begin{problem}[A1984PIQ1l]
{The reading of a speedometer fitted to the front wheel of a bicycle is directly proportional to the angular speed of the wheel. A certain speedometer is correctly calibrated for use with a wheel of diameter $66\textrm{ cm}$ but, by mistake, is fitted to a $60\textrm{ cm}$ wheel. Explain whether the indicated linear speed would be greater or less than the actual linear speed and find the percentage error in the readings. } 
\answer{10}
{\textit{Used with permission from UCLES, A Level Physics, June 1984, Paper 1, Question 1.}}
{The speedometer works by recording the angular speed $\omega$ and converting that into a linear speed using the formula $v=r\omega$. So by attaching it to a bike that has a smaller wheel radius the calculated linear speed will be too great, since the angular speed will be multiplied by $0.66\textrm{ m}$ instead of the actual $0.60\textrm{ m}$. The percentage error is simply given by 
\begin{equation*} \frac{66-60}{60}\cdot100=10\% \end{equation*}
}
\end{problem}