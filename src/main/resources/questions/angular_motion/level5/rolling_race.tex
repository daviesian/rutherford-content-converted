%% ID: rolling_race
%% TITLE: Rolling race
%% TYPE: question
%% QUESTIONTYPE: scq
%% CONCEPTS: com, moi
%% VIDEOS: 
%% LEVEL: 5
%% TOPIC: mechanics/dynamics
%% ORDER: 4

\begin{hint}[Zephyr_rolling_race]
{\exposition{As a truck goes around a sharp corner on a steep mountain road, some of it's cargo falls of the back and rolls down the hill. The truck was transporting barrels of grain, barrels of water (the same size, but much heavier than the grain) and giant spools of copper wire (the same total weight as the water, but much bigger).}
\question{Given that the cargo rolls without slipping, which item of cargo will have the greatest speed once it reaches the bottom?}
\begin{enumerate}
	\item \choice[a]{The wire}
	\item \choice[b]{The water}
	\item \choice[c]{The wire and the water}
	\item \choice[d]{The water and the grain}
	\item \choice[e]{All three will be equal}
\end{enumerate}
  }
{\hint{Assume the slope is uniform and air resistance is negligible. The moment of inertia of a cylinder, mass \vari{m}, radius \vari{r} around it's rotationally symmetric axis is \vari{\frac{m r^2}{2}}}}
{Written by Zephyr Penoyre for the RSPP.}
{\answer{The correct answer is (e).}
This problem is best solved using conservation of energy, although a force method could also be used. Initially all the energy is gravitational potential, and if we choose the zero of GPE to be at the bottom of the slope then all the GPE will be converted to kinetic energy, which has contributions due to both linear and rotational motion:
\begin{equation*}
mgh=\frac{1}{2}mv^2+\frac{1}{2}I\omega^2
\end{equation*}
Since the cargo doesn't slip, $v=r\omega$. Using this and the fact the moment of inertia of a cylinder about an axis through its centre of mass is $\frac{1}{2}mr^2$ we find
\begin{align*}
mgh&=\frac{1}{2}mv^2+\frac{1}{2}\left(\frac{1}{2}mr^2\cdot \frac{v^2}{r^2}\right)=\frac{3}{4}mv^2 \\
\Rightarrow v^2&=\frac{4}{3}gh \\
\Rightarrow v&=\sqrt{\frac{4}{3}gh}
\end{align*}
This is independent of both the mass and the radius of the cylinder and thus, as all cargo descends the same height $h$, it will all have the same speed at the foot of the mountain
}
\end{hint}