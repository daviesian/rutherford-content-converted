%% ID: starting_w_basics
%% TITLE: Starting with the Basics
%% TYPE: question
%% QUESTIONTYPE: scq
%% CONCEPTS: angular_circular, newtonii
%% VIDEOS: AO1989PIQ10l.mp4
%% LEVEL: 4
%% TOPIC: mechanics/circular
%% ORDER: 1
\begin{problem}[AO1989PIQ10l] 
{When a particle travels in a circle of radius $r$ with constant speed $v$, its acceleration is:
\begin{enumerate}
	\item $\frac{v^{2}}{r}$ towards the centre. \answer
	\item $\frac{v^{2}}{r}$ away from the centre.
	\item zero
	\item $rv^{2}$ towards the centre.
	\item $rv^{2}$ away from the centre.
\end{enumerate}}
{\textit{Used with permission from UCLES, AO Level Physics, June 1989, Paper 1, Question 10.}}
{The correct answer is (a). The centripetal force required to keep an object moving at a constant velocity in a circle is $\frac{mv^{2}}{r}$ towards the centre and so Newton's Second Law, $\vtr{F} = m\vtr{a}$ says that the acceleration must therefore be $\frac{v^{2}}{r}$, towards the centre also since $\vtr{F}$ and $\vtr{a}$ must lie in the same direction.}
\end{problem}