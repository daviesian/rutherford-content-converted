%% ID: moi_objects
%% TITLE: An object's Moment of Inertia
%% TYPE: question
%% QUESTIONTYPE: numeric
%% CONCEPTS: moi, integration
%% VIDEOS: 
%% LEVEL: 5
%% TOPIC: mechanics/dynamics
%% ORDER: 9

\begin{hint}[Ed_MoI_One]
{\exposition{Calculate the moments of inertia of the following objects:}
\begin{enumerate}
\item \question[a]{7 point masses of mass \vari{m}, situated at the vertices and centre of a regular hexagon of side length \vari{l}, about the axis perpendicular to the plane of the hexagon and passing through its centre.}
\item  \question[b]{A uniform rod of mass \vari{m} and length \vari{l}, about its end.}
\item  \question[c]{A uniform rod of mass \vari{m} and length \vari{l}, about its centre of mass.}
\item  \question[d]{The same situation as in part a), but now with 6 rods of mass \vari{m} and length \vari{l} connecting the mass at the centre of the hexagon to the other 6 masses.}
\end{enumerate}
}
{
\begin{enumerate}
\setcounter{enumi}{1}
\item \hint[a]{You need to consider an infinitesimal element, of width \vari{\d x} and a distance \vari{x} from one end, and integrate along the rod. %definitely need video instruction first}
\setcounter{enumi}{2}
\item \hint[b]{This can be done from first principles using integration, or by using the parallel axis theorem and the result from part b).}
\end{enumerate}
}
{\textit{Created for the Rutherford School Physics Project by ES.}}
{\answer[a]{\valuedef{I}{6ml^2}{}}
\answer[b]{\valuedef{I_{\text{CoM}}}{\frac{1}{3}ml^2}{}}
\answer[c]{\valuedef{I_{\text{CoM}}}{\frac{1}{12}ml^2}{}}
\answer[d]{\valuedef{I}{9ml^2}{}}
\begin{enumerate}
\item Each of the 6 masses at the vertices are a distance $l$ away from the CoM and the mass at the centre has $r=0$ so makes no contribution. Since $I=ml^2$ for each mass, the total is $I=6ml^2$. 
(The central mass gives no contribution as it has no perpendicular distance from the point of rotation)
\item The rod can be modelled as in Figure \ref{fig:AngularMotion_rod}
\begin{figure}[h]
\centering
\includegraphics[width=7cm]{AngularMotion_rod}
\caption{}
\label{fig:AngularMotion_rod}
\end{figure}
\\
$I=\int x^2dm$ and $dm=\rho dx$ for mass per unit length $\rho$,  therefore
\begin{equation*}
I=\int^l_0\rho x^2dx=\rho[x^3/3]^l_0=\rho l\cdot l^2/3=\frac{1}{3}ml^2
\end{equation*}
\item Either repeat the integration with different limits or use the parallel axis theorem. First the integration method:
\begin{figure}[h] 
\centering
\includegraphics[width=7cm]{AngularMotion_rodtwo}
\caption{}
\label{fig:AngularMotion_rodtwo}
\end{figure}
\\
Figure \ref{fig:AngularMotion_rodtwo} shows the new co-ordinates for the problem, with a general element $dx$ a distance $x$ from the origin. Let the rod have mass per unit length $\rho$ then 
\begin{align*}
\operatorname{d}\!m&=\rho \operatorname{d}\!x \\
\Rightarrow I_{\textrm{CoM}}&=\int_{-l/2}^{l/2}x^2\operatorname{d}\!m=\int_{-l/2}^{l/2}x^2\rho\operatorname{d}\!x \\
&=\rho[x^3/3]_{-l/2}^{l/2}=2\rho(l/2)^3/3 \\
&=\frac{1}{12}\rho x^3
\end{align*}
and finally since $m=\rho l$
\begin{align*}
I_{\textrm{CoM}}=\frac{1}{12}ml^2
\end{align*}
The other way of doing this problem would be to use the parallel axis theorem, so that
\begin{equation*}
I_{\textrm{end}}=I_{\textrm{CoM}}+md^2 
\end{equation*}
where $d$ is the distance between the two parallel axes. Here $d=l/2$ so
\begin{equation*}
I_{\textrm{CoM}}=I_{\textrm{end}}-ml^2/4
\end{equation*}
This method is slightly quicker since we just calculated $I_{\textrm{end}}$ to be $\frac{1}{3}ml^2$, therefore
\begin{align*}
I_{\textrm{CoM}}=\frac{1}{3}ml^2-\frac{1}{4}ml^2=\frac{1}{12}ml^2 
\end{align*}
as before.
\item Just add the contribution from the six rods, all rotating around their end points, so 
\begin{equation*}
I=6ml^2+6\cdot\frac{1}{3}ml^2=9ml^2
\end{equation*}
\end{enumerate}
}
\end{hint}