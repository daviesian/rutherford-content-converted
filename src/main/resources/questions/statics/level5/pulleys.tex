%% ID: pulleys
%% TITLE: Pulley system
%% TYPE: question
%% QUESTIONTYPE:  scq
%% CONCEPTS: forces, vectors1, newtoni, newtonii, newtoniii
%% VIDEOS: 
%% LEVEL: 5
%% TOPIC: mechanics/statics
%% ORDER: 1

\begin{problem}[Zephyr_Pulleys]
{\exposition{A simple pulley system consists of a light inextensible string and two pulleys, the higher attached to a solid ceiling and with a weight hung off the other as shown:
\begin{figure}[h]
	\centering
	\includegraphics[width=0.2\textwidth]{../../../figures/Statics_Pulleys.svg}
	\caption{}
	\label{fig:Statics_banner}
\end{figure}}
\nl
\question{The force $F$ needed here to support weight $W$ is $\frac{W}{2}$, what force would be needed if a third pulley was added to the system?}
\begin{enumerate}
\item \choice[a]{\vari{W/2}}
\item \choice[b]{\vari{W/4}}
\item \choice[c]{\vari{W/3}}
\item \choice[d]{\vari{W/\sqrt{2}}}
\item \choice[e]{\vari{W}}
\end{enumerate}
}
{Written by Zephyr Penoyre for the RSPP}
{\answer{The correct answer is (c).}
 In the two pulley system the weight W is supported by two lengths of string and tension is constant throughout the string. Thus the tension of the string $T = \frac{W}{2}$ is equal to the force $F$. If a third pulley was added the weight would then be spread between three sections of string all with equal tension.
Therefore $F = T = \frac{W}{3}$.
}
\end{problem}