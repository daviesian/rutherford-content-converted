\begin{hint}[Tripos_Angular_Rod]
{
A uniform rod of length $l$ and mass $2m$ rests on a smooth horizontal table. A point mass $m$ moving at right angles to the rod with an initial velocity $u$ collides with one end of the rod and sticks to it. Given $l=1$m, $m=2$kg and $u=3$ms\sup{-1}, find:
\begin{enumerate}
\item the angular speed of the system after the collision.
\item the fraction of the length, from the left hand end, of the position of the point on the rod that remains stationary immediately after the collision.
\item the total change in the kinetic energy of the system as a result of the collision. 
\end{enumerate}
}
{
\begin{enumerate}
\item The rod will rotate about its centre of mass; think carefully about where this is and what the appropriate moment of inertia will be. (The moment of inertia of a rod mass $m$ length $l$ about its centre of mass was calculated in the first problem sheet).  
\item Remember to consider both the linear and rotational motion of the rod.  
\end{enumerate}
}
{%\textit{Used/Adapted with permission from UCLES, A/O Level Physics/Maths/Physical Science, June/November 198, Paper, Question.}
}
{
\begin{enumerate}
\item As always a diagram is helpful. Figure \ref{fig:AngularMotion_rodcoll} shows the situation before the collision:
\begin{figure}[h] 
\centering
\includegraphics[width=9cm]{AngularMotion_rodcoll}
\caption{}
\label{fig:AngularMotion_rodcoll}
\end{figure}
\\
and Figure \ref{fig:AngularMotion_rodcolltwo} shows what has happened after the mass sticks to the rod:
\begin{figure}[h] 
\centering
\includegraphics[width=9cm]{AngularMotion_rodcolltwo}
\caption{}
\label{fig:AngularMotion_rodcolltwo}
\end{figure}
\\
The centre of mass (CoM) moves with velocity $\vtr{v}$ and the composite body rotates about its CoM with angular velocity $\vtr{\omega}$. Notice how the CoM has moved. It will be necessary to know where the new CoM is:
\begin{align*}
CoM=\frac{\Sigma \textrm{ mass}\times \textrm{distance}}{\Sigma \textrm{ mass}}=\frac{2m\cdot l/2+m\cdot l}{2m+m}=\frac{2}{3}l
\end{align*}
relative to the left hand end of the rod, which corresponds to a distance $l/6$ from the old CoM, of the rod alone. 
We now need to conserve both linear and angular motion. Linearly
\begin{align*}
mu=3mv \\
\Rightarrow v=u/3
\end{align*}
and angularly, about the new CoM
\begin{align*}
mu\cdot l/3=I\omega \\
\Rightarrow \omega=\frac{mul}{3I}
\end{align*}
So to determine the new angular speed we need to know the new moment of inertia (MoI). The MoI of a rod mass $m$ length $l$ about its CoM is $\frac{1}{12}ml^2$, as calculated in an earlier problem sheet. So the MoI a of a rod of mass $2m$ length $l$ with an additional mass $m$ on the end, call it $I'$, about the centre of the rod (not the CoM) is
\begin{align*}
I'=\frac{1}{12}(2m)l^2+m\left(\frac{l}{2}\right)^2=\frac{5}{12}ml^2
\end{align*}
We then need to find the MoI about the CoM of this body, which we established is a distance $l/6$ away from the centre of the rod. Using the parallel axis theorem, and with $I$ denoting the MoI about the CoM:
\begin{align*}
I'&=I+3m\left(\frac{l}{6}\right)^2 \\
\Rightarrow I&=\frac{5}{12}ml^2-3m\frac{l^2}{36}=\frac{1}{3}ml^2
\end{align*}
We can now finally work out the new angular speed of the system:
\begin{align*}
\omega=\frac{mul}{3I}&=\frac{mul}{3}\cdot\frac{3}{ml^2} \\
\Rightarrow \omega&=\frac{u}{l}
\end{align*}
Thus using $l=1$ and $u=3$ gives $\omega = 3$ rad s\sup{-1}

\item We know that the CoM of the rod is moving upwards with speed $v=u/3$. Therefore we need to find the point on the rod where the linear speed due to the rotation about the CoM is $u/3$ in the opposite direction. Figure \ref{fig:AngularMotion_rodcollthree} shows how we will do this calculation. 
\begin{figure}[h] 
\centering
\includegraphics[width=9cm]{AngularMotion_rodcollthree}
\caption{}
\label{fig:AngularMotion_rodcollthree}
\end{figure}
The point at a distance $x$ away from the CoM on the left hand side will have a speed $u/3$ due to the motion of the CoM and a speed of $\omega x$ in the opposite direction due to rotation. So equate the two to find the point where they cancel (using $\omega=u/l$):
\begin{align*}
\omega x&=\frac{u}{3} \\
\Rightarrow x&=\frac{u}{3}\cdot\frac{l}{u} \\
\Rightarrow x&=\frac{l}{3}
\end{align*}
So the point a one third of the length away from the CoM on the left (and therefore a  $\frac{1}{3}l$ from the left hand end of the rod) is stationary immediately after the collision. 
\item Initially the kinetic energy, $E_0$, was given by
\begin{align*}
E_0=\frac{1}{2}mu^2
\end{align*}
and the final kinetic energy $E_1$ is 
\begin{align*}
E_1&=\frac{1}{2}3mv^2+\frac{1}{2}I\omega^2 \\
&=\frac{1}{2}3m\left(\frac{u}{3}\right)^2+\frac{1}{2}\frac{ml^2}{3}\left(\frac{u}{l}\right)^2
&=\frac{1}{3}mu^2
\end{align*}
so the change in kinetic energy $\Delta E$ is
\begin{align*}
\Delta E=E_1-E_0=-\frac{1}{6}mu^2
\end{align*} 
Thus with $m = 2$ and $l = 3$, $\Delta E = - 3$ J. The minus sign indicates that energy is lost overall. 
\end{enumerate}
}
\end{hint}