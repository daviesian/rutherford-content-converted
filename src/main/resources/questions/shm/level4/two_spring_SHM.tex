%% ID: two_spring_SHM
%% TITLE: Two Spring SHM
%% TYPE: question
%% QUESTIONTYPE: scq
%% CONCEPTS: forces, energy, hooke, shm
%% VIDEOS: 
%% LEVEL: 4
%% TOPIC: mechanics/shm
%% ORDER: 4

\begin{problem}[A1981PIIQ16a] %Diagram removed
{\exposition{A mass m on a smooth horizontal table is attached by two identical light springs to two fixed supports. Each spring has a natural length equal to half the distance between the supports. The mass executes linear simple harmonic motion of amplitude \vari{a} and period \vari{T}.} \question{What is the energy associated with this simple harmonic motion?
}\begin{enumerate}
	\item \choice[a]{$\frac{2\pi ma^2}{T^2}$}
	\item \choice[a]{$\frac{2\pi m^2a^2}{T}$}
	\item \choice[a]{$\frac{\pi^2ma^2}{T^2}$}
	\item \choice[a]{$\frac{2\pi^2ma^2}{T^2}$}\correct
	\item \choice[a]{$\frac{4\pi^2ma^2}{T^2}$}
\end{enumerate}
}
{\textit{Used with permission from UCLES, A Level Physics, June 1981, Paper 2, Question 1}}
{\answer{The correct answer is (d).} The total energy of the system can be found easily when the mass is at the equilibrium position, when all of the energy is in the form of kinetic energy and none is in the form of elastic potential energy. The speed at this position is at a maximum, where $v = -a\omega\sin{\omega t}$. \vari{v\s{max}} is where $\sin{\omega t}$ is at a minimum, and the minimum value of the $\sin$ function is $-1$; therefore $v\s{max} = a\omega$. $E = \frac{1}{2}mv\s{max}^{2} = \frac{1}{2}ma^{2}\omega^{2}$, and $\omega = \frac{2\pi}{T}$, so $E = \frac{2\pi^{2}ma^{2}}{T^{2}}$. This answer can also be found by considering the point of maximum displacement, when all of the energy is in the form of elastic potential energy, so $E = \frac{1}{2}\left(2k\right)a^2$ (as there are two springs). Next recall that $\omega = \sqrt{\frac{2k}{m}}$ (again, not forgetting that there are two springs). \vari{\omega} also equals $\frac{2\pi}{T}$, so $k = \frac{2\pi^2m}{T^2}$. Substitute this back in to obtain the correct answer.
}
\end{problem}