%% ID: IntA1988PIQ2p
%% TITLE: A Framework
%% TYPE: question
%% QUESTIONTYPE:  numerical
%% CONCEPTS: 
%% VIDEOS: 
%% LEVEL: 2
%% TOPIC: mechanics/statics
%% ORDER: 8

\begin{problem}[IntA1988PIQ2p] %frameworks
%no longer multiple choice
{\exposition{Four light, hinged rods are connected together in order to make a framework to support a load. Rod AB is hinged to a wall at A. Rod CD is half the length of AB and is hinged to the same wall at C, which is directly below A. The other two rods (which are the same length as AB) are smoothly hinged at each end; one to the points A and D, and the other to the points B and D.}\question[a]{Draw a diagram, labelling the rods according to the text.} \question[b]{What is the value of the angle CDB?}
}
{\textit{Adapted with permission from UCLES, A Level Physics, November 1988, Paper I, Question 2}}
{\answer[a]{\begin{figure} [h]
	\centering
	\includegraphics[width=0.4\textwidth]{Statics_framework_angles}
	\caption{}
	\label{fig:Statics_framework_angles}
\end{figure}}
The diagram in Figure \ref{fig:Statics_framework_angles} shows the framework that has been set up. The rods AB, AD and BD are all the same length, so triangle ABD is equilateral; this means that all of its internal angles are $60^{\circ}$. \answer[b]{Angle CDB is  $120^{\circ}$.}

}
\end{problem}