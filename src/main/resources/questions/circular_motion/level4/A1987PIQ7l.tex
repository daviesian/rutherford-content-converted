%% ID: the_minute_hand
%% TITLE: The Minute Hand
%% TYPE: question
%% QUESTIONTYPE: scq
%% CONCEPTS: angular_ciruclar
%% VIDEOS: A1987PIQ7l.mp4
%% LEVEL: 4
%% TOPIC: mechanics/circular
%% ORDER: 3

\begin{problem}[A1987PIQ7l] 
{The minute hand of a large clock is 3.0 m long. What is its mean angular speed?
\begin{enumerate}
	\item $1.4 \times 10^{-4} \textrm{ rad s}^{-1}$
	\item $1.7 \times 10^{-3} \textrm{ rad s}^{-1}$
	\item $5.2 \times 10^{-3} \textrm{ rad s}^{-1}$
	\item $1.0 \times 10^{-1} \textrm{ rad s}^{-1}$ \answer
	\item $3.0 \times 10^{-1} \textrm{ rad s}^{-1}$
\end{enumerate}}
{\textit{Used with permission from UCLES, A Level Physics, June 1987, Paper 1, Question 7.}}
{The correct answer is (d). This follows from the fact that there are $2\pi$ radians in a full circle, and the hand goes round every 60 seconds. Hence the mean angular speed is given by $\omega = \frac{2\pi}{60} = 0.10472 \textrm{ rad s}^{-1} \approx 1.0 \times 10^{-1} \textrm{ rad s}^{-1}$. We can also see some of the information is irrelevant as the radius of the hand has no bearing on its angular speed.}
\end{problem}