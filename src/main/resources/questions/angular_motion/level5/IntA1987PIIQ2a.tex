\begin{problem}[IntA1987PIIQ2a] 
{Halley's comet moves in an elliptical orbit about the Sun; no external torque acts on the system. The orbit has its closest approach to the Sun at $8.8\times 10^{10}\textrm{ m}$. The furthest point is a distance $5.3\times 10^{12}\textrm{ m}$ away from the Sun, and it's speed at that point is  $913\textrm{ ms}^{-1}$; what is the highest speed the comet will attain in it's orbit?
}
{\textit{Used with permission from UCLES, A Level Physics, November 1987, Paper 2, Question 2.}}
{
Angular momentum is conserved since no external torque acts on the system. At the closest point the magnitude of the angular momentum is 
\begin{equation*}
J=I_c\omega_c=mr_c^2\cdot \frac{v_c}{r_c}=mr_cv_c
\end{equation*}
where $r_c$ is the distance of the comet from the Sun at the point of closest approach, and $v_c$ is the speed at that point. The magnitude of the angular momentum must have the same value when the comet is at the furthest point so 
\begin{align*}
J=mr_fv_f \\
\end{align*}
where $r_f$ is the distance of the comet from the Sun at the furthest point away, and $v_f$ is its speed there. 
\begin{align*}
\Rightarrow r_fv_f=r_cv_c \\
\Rightarrow v_c=\frac{r_f v_f}{r_c}
\end{align*}
The numerical answer is then $55,000\textrm{ ms}^{-1}$.
}
\end{problem}