%% ID: walking_up_hill
%% TITLE: Walking up a Hill
%% TYPE: question
%% QUESTIONTYPE: numeric
%% CONCEPTS: energy, trigonometry, vectors, resolving_vectors
%% VIDEOS: O1980PIQ2a_work.mov, O1980PIQ2a_1in5slope.mov
%% LEVEL: 2
%% TOPIC: mechanics/dynamics
%% ORDER: 7

\begin{problem}[O1980PIQ2a]%Needs Work, and the ability to do geometry/trigonometry.
{\exposition{A person of mass \quantity{80}{kg} walks for \quantity{102}{m} along a path going up a hill of uniform gradient 1 in 5.}\question[a]{Calculate the work done against gravity by this person.} \question[b]{What happens to this energy?} Write your answer to one decimal place.

\stress{Hint:} \hinta{A gradient of 1 in 5 means that for every \quantity{5}{m} horizontally, \quantity{1}{m} is gained in vertical height. It is also called a slope of \quantity{20}{\%}.}}
{\stress{Adapted with permission from UCLES, O Level Physics, June 1980, Paper 1, Question 2.}}
{The work done against gravity is dependant only on the vertical height gained by the person. Work done, \vari{W} is the force times the distance moved in the direction of the force: \vari{W} = \valuedef{\vtr{F}\cdot\vtr{x}}{Fh}{}, where \vari{\vtr{x}} is the displacement vector up the slope, and \vari{h} the vertical height gained. \\

A slope of 1 in 5 gains 1 m of height for every 5 m of horizontal distance covered, and so the length of the hypotenuse will give the distance along the hill covered for every 1 m of height gained. Simple Pythagoras gives that the hypotenuse of the triangle is \quantity{\sqrt{26}}{m} = \quantity{5.1}{m}, and so for every \quantity{5.1}{m} walked, \quantity{1}{m} in height is gained. This gives a total height of \valuedef{\frac{102}{\sqrt{26}}}{20.0}{m}.\\

Work done is the force times this distance in the direction the force is acting, \vari{W} = \valuedef{\vtr{F}\cdot\vtr{x}}{Fd}{}, where \vari{\vtr{x}} is the displacement and \vari{d} the distance in the direction of \vari{\vtr{F}}. The force here is the weight, \vari{mg}, and so the total work is \answer[a]{\vari{W} = \vari{mgh} = \valuedef{(80)(9.8)(20)}{15.7}{kJ}.} \answer[b]{This energy is transformed into gravitational potential energy.}
}
\end{problem}