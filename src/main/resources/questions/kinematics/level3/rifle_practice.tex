%% ID: rifle_practice
%% TITLE: Rifle Practice
%% TYPE: question
%% QUESTIONTYPE: scq
%% CONCEPTS: eq_of_motions_diff, forces, newtonii, vectors1
%% VIDEOS: 
%% LEVEL: 3
%% TOPIC: mechanics/kinematics
%% ORDER: 2


\begin{problem}[IntA1982PIIQ4a] %Diagram removed
{\exposition{When a rifle is fired horizontally at a target $P$ on a screen at a range of \quantity{25}{m}, the bullet strikes the screen at a point \quantity{5.0}{mm} below $P$. The screen is now moved to a distance \quantity{50}{m} and the rifle again fired horizontally at $P$ in its new position.} \question{Assuming that air resistance may be neglected, what is the new distance below $P$ at which the screen would now be struck?}
\begin{enumerate}
	\item \choice[a]{$5\sqrt{2}$ mm}
	\item \choice[b]{10 mm}
	\item \choice[c]{15 mm}
	\item \choice[d]{20 mm}\correct
	\item \choice[e]{25 mm}
\end{enumerate}
}
{\textit{Used with permission from UCLES, A Level Physics, November 1982, Paper 2, Question 2.}}
{\answer{The correct answer is (d).} The SUVAT equation $s = ut + \frac{1}{2}at^{2}$ can be applied to the vertical height through which the bullet drops. The initial vertical speed \vari{u} of the bullet is zero, so $s = \frac{1}{2}at^{2}$; therefore $s \propto t^{2}$. The horizontal speed of the bullet is constant, so if the screen is moved twice as far away, then the bullet takes twice as long to reach the screen. This means that \vari{t} doubles, so using the proportional relationship, the height by which the bullet drops quadruples from \quantity{5.0}{mm} to \quantity{20}{mm} .
}
\end{problem}