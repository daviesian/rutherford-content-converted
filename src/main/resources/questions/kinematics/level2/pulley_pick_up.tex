%% ID: pulley_pick_up
%% TITLE: Pulley picking up a mass
%% TYPE: question
%% QUESTIONTYPE: numeric
%% CONCEPTS: forces, newtonii, eq_of_motions_diff
%% VIDEOS: 
%% LEVEL: 2
%% TOPIC: mechanics/kinematics
%% ORDER: 9


\begin{problem}[HSC1930MIIIQ4p] % a standard question with a twist - picking up another mass
 {\exposition{Two masses $3M$ and $2M$ are suspended by a light string passing over a smooth peg. The system starts from rest and at the end of $t$ seconds the mass $2M$ picks up a mass $4M$ which was previously at rest.} \question{What is the time taken, $T$, for the mass $3M$ to come to rest (as a proportion of the time, $t$)?}}
{\textit{Used with permission from UCLES, Higher School Certificate Mathematics, June 1930, Paper 3, Question 4.}}
{See Figure \ref{fig:Kinematics_pickup_mass_SUVAT_1} for reference.
The initial net force is
\begin{align*}
3Mg-2Mg&=5Ma_1 \\
\Rightarrow a_1&=\frac{g}{5}
\end{align*}

\begin{figure}[h]
	\centering
	\includegraphics[width=8cm]{Kinematics_pickup_mass_SUVAT_1}
	\caption{}
	\label{fig:Kinematics_pickup_mass_SUVAT_1}
\end{figure}

So the velocity at time $t$ is $a_1t=\frac{gt}{5}$. Once the second mass is picked up, the acceleration is in the opposite direction:
\begin{align*}
3Mg-6Mg&=9Ma_2 \\
\Rightarrow a_2&=-\frac{g}{3}
\end{align*}
Then using $v=u+at$ and setting $v=0$ since we want the time taken for the bodies to come to rest $T$:
0&=\frac{gt}{5}-\frac{gT}{3} \\
\answer{T&=\frac{3t}{5}}
as required. 
}
\end{problem}