%% ID: shm_simple_scale
%% TITLE: SHM of a simple scale
%% TYPE: question
%% QUESTIONTYPE: numerical
%% CONCEPTS: forces, hooke, shm
%% VIDEOS: 
%% LEVEL: 4
%% TOPIC: mechanics/shm
%% ORDER: 6

%Physics: resolving forces, NII, SHM, omega, period, hooke's law, CoE
%Maths:
\begin{problem}[A1951PIQ3p]
{\exposition{The upper end of a light vertical spring is fixed and its lower end is hanging freely. A scale pan of mass \quantity{20}{g} is then hung from the free end.}
\begin{enumerate}

	\item \question[a]{Prove that the resulting vertical motion is simple harmonic, and calculate the period and amplitude if the spring constant is \quantity{10}{g\,cm\sup{-1}}.}

	\item \exposition{The oscillating pan is held at the equilibrium and then released so that it is at rest. A mass of \quantity{20}{g} is introduced into the scale pan so that it once again moves with simple harmonic motion.} \question[b]{Find the new period and amplitude.}
	
\end{enumerate}
}
{\textit{Used with permission from UCLES, A Level Physics, June 1951, Paper 1, Question 3.}}
{\begin{enumerate}

	\item Let the spring have a natural length \vari{l}. When the scale pan in equilibrium, let the spring have an extension \vari{x_0} as shown in Figure \ref{fig:SHM_scale_pan_empty}. As there is no resultant force acting at this point we can resolve forces vertically on the pan to give $0.02g = kx_0$.
\begin{figure}
	\centering
	\includegraphics[width=0.2\textwidth]{SHM_scale_pan_empty}
	\caption{}
	\label{fig:SHM_scale_pan_empty}
\end{figure}

\nl At any point in the motion let spring have an extra extension \vari{x}, as shown in Figure \ref{fig:SHM_scale_pan_full}. Its acceleration acting downwards is given by \vari{\ddot{x}}. Applying $F = ma$ to the pan:
\begin{figure}
	\centering
	\includegraphics[width=0.2\textwidth]{SHM_scale_pan_full}
	\caption{}
	\label{fig:SHM_scale_pan_full}
\end{figure}
\begin{equation*}
0.02g - kx - kx_0 = 0.02\ddot{x}
\end{equation*}
Recalling that $kx_0 = 0.02g$:
\begin{align*}
- kx&= 0.02\ddot{x} \\
\Rightarrow \ddot{x}&=-\frac{k}{0.02}x
\end{align*}
This gives us $\ddot{x}\propto -x$ as required for SHM.

\nl Angular frequency $\omega = \sqrt{\frac{k}{m}}$, and period $T = \frac{2\pi}{\omega}$, so $T = 2\pi\sqrt{\frac{m}{k}}$. If \value{k}{10}{g\,cm\sup{-1}} $=$ \quantity{1}{kg\,m\sup{-1}}, then $T = 2\pi\sqrt{\frac{0.02}{1}} = 0.889$ s.

\nl The pan starts at a height with the spring unstretched, and then oscillates about a point where the extension is \vari{x_0}. The amplitude of the motion is therefore equal to this equilibrium extension. $0.02g = kx_0$, so $x_0 = \frac{0.02g}{k} = 0.196$ m.

\answer[a]{The period of the oscillation is \valuedef{T}{0.89}{s} and the amplitude of the oscillation is \valuedef{A}{0.20}{m}.}

	\item The period can again be found using the expression $T = 2\pi\sqrt{\frac{m}{k}}$. \valuedef{m}{0.04}{kg}, so \valuedef{T}{1.26}{s}.
	
\nl To find the new amplitude, consider the equilibrium point of the new motion. Let the extension of the spring at this point be \vari{x_1}. Resolving forces vertically at this point gives $0.04g = kx_1$, so $x_1 = 0.392$ m. The pan starts from a point where the extension of the spring is \vari{x_0}, so the new amplitude is given by $x_1-x_0 = 0.196$ m - the same as the previous amplitude.

\answer[b]{The period of the oscillation is \valuedef{T}{1.26}{s} and the amplitude of the oscillation is \valuedef{A}{0.20}{m}, the same as before.}


\end{enumerate}
}
\end{problem}