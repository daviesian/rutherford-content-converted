%% ID: pendulum_lift
%% TITLE: Pendulum in a Lift
%% TYPE: question
%% QUESTIONTYPE: scq
%% CONCEPTS: shm
%% VIDEOS: 
%% LEVEL: 4
%% TOPIC: mechanics/shm
%% ORDER: 3

%Would like to add something about relative motion or frames of reference into concepts here
\begin{problem}[IntA1986PIQ8l]
{\exposition{A simple pendulum suspended from the ceiling of a stationary lift has period \vari{T_0}. When the lift descends at steady speed the period is \vari{T_1}, and when it descends with constant downward acceleration (which is less than \vari{g}) the period is \vari{T_2}.} \question{Which one of the following is correct?}
\begin{enumerate}
	\item \choice[a]{$T_0 = T_1 = T_2$}
	\item  \choice[b]{$T_0 = T_1 < T_2$}\correct
	\item  \choice[c]{$T_0 = T_1 > T_2$}
	\item  \choice[d]{$T_0 < T_1 < T_2$}
	\item  \choice[e]{$T_0 > T_1 > T_2$}
\end{enumerate}
}
{\textit{Used with permission from UCLES, A Level Physics, November 1986, Paper 1, Question 8.}}
{\answer{The correct answer is (b)}. $T_0 = T_1$ as physical effects are unchanged in all inertial frames of reference, i.e. when the system is not accelerating. For \vari{T_2}, the entire system is accelerating downwards, so the pendulum's effective acceleration due to gravity is decreased. We know that $T \propto \sqrt{\frac{l}{g}}$, so if \vari{g} effectively decreases, then \vari{T} must increase. This becomes clear when you consider the case where the lift accelerates downwards with \vari{g}: the pendulum appears weightless, so \vari{T} decreases to 0.
}
\end{problem}