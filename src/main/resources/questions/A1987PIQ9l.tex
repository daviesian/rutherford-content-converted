\begin{problem}[A1987PIQ9l] 
{A mass, $m$ of 0.050 kg is attached to one end of a piece of elastic of unstretched length, $l = 0.50 \textrm{ m}$. The force constant, $k$ of the elastic (i.e. the force required to produce unit extension) is $40 \textrm{ N m}^{-1}$. The mass is rotated steadily on a smooth table around a fixed point in a horizontal circle of radius $r =0.70 \textrm{ m}$.
What is the approximate speed of the mass?
\begin{enumerate}
	\item $11 \textrm{ ms}^{-1}$
	\item $15 \textrm{ ms}^{-1}$
	\item $20 \textrm{ ms}^{-1}$
	\item $24 \textrm{ ms}^{-1}$
	\item $28 \textrm{ ms}^{-1}$
\end{enumerate}}
{\textit{Adapted with permission from UCLES, A Level Physics, June 1987, Paper 1, Question 9.}}
{The correct answer is (a). The tension from the extension, $T = kx$ can be equated to the force from the radial acceleration, $a\s{R} = m\frac{v^{2}}{r}$, as these are the only two forces acting radially during the motion of $m$. The velocity can be calculated from: $\frac{mv^{2}}{r} = kx$. Since the extension, $x$ is $(0.70 - 0.50) \textrm{ m} = 0.20 \textrm{ m}$, $v = \sqrt{\frac{kxr}{m}} = \sqrt{\frac{(40)(0.2)(0.7)}{0.05}} \textrm{ ms}^{-1}= 10.58 \textrm{ ms}^{-1} \approx 11 \textrm{ ms}^{-1}$.}
\end{problem}