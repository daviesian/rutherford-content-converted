%% ID: A1986PIQ1l
%% TITLE: Size of a combined force
%% TYPE: question
%% QUESTIONTYPE:  scq
%% CONCEPTS: forces, vectors1
%% VIDEOS: 
%% LEVEL: 2
%% TOPIC: mechanics/statics
%% ORDER: 2

\begin{problem}[A1986PIQ1l]
{\exposition{Forces of 4 N and 6 N act at a point.} \question{Which one of the following could \textbf{not} be the magnitude of their resultant?}
\begin{enumerate}
	\item \choice[a]{1 N}\correct
	\item \choice[b]{4 N}
	\item \choice[c]{6 N}
	\item \choice[d]{8 N}
	\item \choice[e]{10 N}
\end{enumerate}
}
{\textit{Used with permission from UCLES, A Level Physics, June 1986, Paper 1, Question 1.}}
{\query{The correct answer is (a). The smallest possible value that the resultant force can take is 2 N, where the two forces act in opposite directions.} Resultants of 4 N, 6 N and 8 N are all possible and can be demonstrated using vector triangles; 10 N is the largest possible resultant, where the two forces act in the same direction.
}
\end{problem}