%% ID: geostationary_orbit
%% TITLE: Geostationary Orbit
%% TYPE: question
%% QUESTIONTYPE: numeric
%% CONCEPTS: angular_circular, energy, vectors
%% VIDEOS: A1989PSIIQ3.mp4
%% LEVEL: 4
%% TOPIC: mechanics/circular
%% ORDER: 7

\begin{problem}[A1989PSIIQ3l] 
{A satellite is to be placed in a circular orbit around the Earth. Use the information and data below to calculate the required radius of the orbit if the satellite is in a geostationary orbit (remains above the same point on the equator). \\ The gravitational force \vari{F_{A}} between the satellite and the Earth is in the inward radial direction and its magnitude is given by the equation 
\begin{equation*}
F_A=GMm/R^2
\end{equation*}
where \vari{G} is the gravitational constant; \vari{M} and \vari{m} are the masses of the Earth and the satellite respectively; and \vari{R} is the radius of the orbit. Please give your answer to two significant figures.

\emph{Hint:} 
\value{G}{6.67\times 10^{-11}}{m\sup3\,kg\sup{-1}\,s\sup{-2}}\nl
\value{M}{5.97\times 10^{24}}{kg}
\end{equation*}
} {\stress{Used with permission from UCLES, A Level Physical Science, June 1989, Paper 2, Question 3.}
}{
Use the requirement that in circular motion the magnitude of the centripetal force \value{F_{c}}{mR\omega^2}{} (this is equivalent to saying that \value{F_{c}}{mv^2/R}{} since \value{v}{r\omega}{} in circular motion) must be equal to the magnitude of the gravitational attraction to obtain \begin{equation*}
mR\omega^2=GMm/R^2
\end{equation*}
 and then rearrange to show that
 \begin{equation*}
{R={(GM/w^2)}^{1/3}.}
\end{equation*} 
Finally use \vari{\omega} corresponding to the angular speed of Earth's rotation about its axis to find the numerical answer, which is \quantity{4.2\times 10^7}{m}.}
\end{problem}