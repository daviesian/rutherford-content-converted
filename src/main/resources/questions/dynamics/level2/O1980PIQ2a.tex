%% ID: walking_up_hill
%% TITLE: Walking up a Hill
%% TYPE: question
%% QUESTIONTYPE: numeric
%% CONCEPTS: energy, trigonometry, vectors, resolving_vectors
%% VIDEOS: O1980PIQ2a_work.mov, O1980PIQ2a_1in5slope.mov
%% LEVEL: 2
%% TOPIC: mechanics/dynamics
%% ORDER: 7

\begin{problem}[O1980PIQ2a]%Needs Work, and the ability to do geometry/trigonometry.
{Calculate the work done against gravity by a person of mass \quantity{80}{kg} in walking for \quantity{102}{m} along a path going up a hill of uniform gradient 1 in 5. What happens to this energy? Write your answer to one decimal place.

\emph{Hint:} A gradient of 1 in 5 means that for every \quantity{5}{m} horizontally, \quantity{1}{m} is gained in vertical height. It is also called a slope of \quantity{20}{\%}.}
{\stress{Adapted with permission from UCLES, O Level Physics, June 1980, Paper 1, Question 2.}}
{The work done against gravity is dependant only on the vertical height gained by the person. Work done, $W$ is the force times the distance moved in the direction of the force: $W = \vtr{F}\cdot\vtr{x} = Fh$, where $\vtr{x}$ is the displacement vector up the slope, and $h$ the vertical height gained. \\

A slope of 1 in 5 gains 1 m of height for every 5 m of horizontal distance covered, and so the length of the hypotenuse will give the distance along the hill covered for every 1 m of height gained. Simple Pythagoras gives that the hypotenuse of the triangle is $\sqrt{26} \textrm{ m} \approx 5.1 \textrm{ m}$, and so for every 5.1 m walked, 1 m in height is gained. This gives a total height of $\frac{102}{\sqrt{26}} \textrm{ m} = 20.0 \textrm{ m}$.\\

Work done is the force times this distance in the direction the force is acting, $W = \vtr{F}\cdot\vtr{x} = Fd$, where $\vtr{x}$ is the displacement and $d$ the distance in the direction of $\vtr{F}$. The force here is the weight, $mg$, and so the total work is $W = mgh = (80)(9.8)(20) \textrm{ J} = 15.7 \textrm{ kJ}$. This energy is transformed into gravitational potential energy.
}
\end{problem}