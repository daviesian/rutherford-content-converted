%% ID: energy
%% QUESTIONS: walking_up_hill, what_goes_up, the_lift, particles_on_string, energy_of_bullet, orbiting_the_moon, geostationary_orbit, powerful_stuff, firing_a_canon, skiers, acceleration_string, three_collisions
%% CONCEPTS: newtonii, eq_of_motion_diff
%% LEVEL: 2
%% TOPIC: mechanics/dynamics
%% TYPE: physics
%% TITLE: Energy
%% ORDER: 70

% This is the template that sets out all of the Problems and produces the Exercise/Solution labels and numbering
% There are two classes of Exercise: "problem" which has a Question and Solution, and "hint" which has a Question, Hint and Solution

% This is the template that sets out all of the Problems and produces the Exercise/Solution labels and numbering
% There are two classes of Exercise: "problem" which has a Question and Solution, and "hint" which has a Question, Hint and Solution

% These are the packages to use in all documents, and the paper size to use:
\documentclass[a4paper,11pt]{article}
\usepackage[usenames,dvipsnames]{xcolor}
\usepackage[margin=1.5cm]{geometry}
\usepackage{amsmath}
\usepackage{amssymb}
\usepackage{color}
\usepackage{graphicx}
\usepackage{graphics}
\usepackage[margin=1.5cm]{geometry}
\usepackage{fancyhdr}
\usepackage{float}
\usepackage{lscape}
\usepackage[font={small},labelfont=bf]{caption}
\usepackage{ifthen}
\usepackage{enumitem}
\usepackage{subcaption}		%Allows grouped figures. The percentage sign after the first \end{subfigure} puts them side by side, omitting it puts one above the other.
\usepackage{graphicx,xcolor} 	%Allows the use of colour in the files
\usepackage{centernot} 		%Puts the / in a not equal to sign in the centre, use as \cnot{...}
\usepackage{comment} 		%Allows \begin{comment} .... \end{comment} to comment out bulk text.
\usepackage{etoolbox}		%Allows the boolean flags and the \toggletrue and \togglefalse commands
\usepackage{cancel}		%Allows the crossing out of terms in maths mode to show they cancel out
\usepackage{wrapfig}

%Packages for font choices
\usepackage{palatino}
\usepackage{mathpazo}


%Then where to find the graphics:
%WARNING -  relative to the TeX file being compiled - NOT this template!
		\graphicspath{{../Diagrams/}{Diagrams/}{./}} %This allows diagrams: {{As a sister folder to Latex}{A subdirectory of LaTeX}{Or just in LaTeX itself}}

% WARNING -  If you want the diagrams to be a sister folder to the LaTeX folder - pdflatex.exe sometimes needs an extra argument to cope with the "../" part; usually it can only cope with subdirectories as opposed to parent ones. If it refuses to compile and says it cannot find the diagrams, either add "--shell-escape" to the start of the arguments of pdflatex, OR move the diagrams to a subdirectory of the one containing the TeX files.
%In TeXworks, to add the extra argument, go to Edit -> Preferences -> Typesetting -> Processing Tools. Click on "pdfLaTeX" -> Edit -> "+" button, then type "--shell-escape" (without quotes) and press the up arrow twice so that it becomes top of the list.


%Then any custom commands written, along with shortcuts and variables:
% This document contains any custom commands, shortcuts and variables needed for the files to compile. It is called by "Problem_Template.tex" and so needs to be in the same directory.

%Defines vectors universally, for ease of editing and consistency.
\newcommand{\vtr}[1] {\mathit{\underline{\boldsymbol{#1}}}}

%Draws a big red box containing the text as in \ALERT{<TEXT HERE>}. For labelling draft copies with important notes.
\def\ALERT#1{\begin{center}\colorbox{red}{\hbox{\textcolor{black}{\textbf{#1}}}}\end{center}}

%Roman-style subscript; removes math-mode font.
\def\s#1{_\textrm{#1} }

%The operators in integrals and derivatives.
\def\d{\operatorname{d}\!}

%The Euler e should be in Roman font.
\def\e{\textrm{e}}

%The Rutherford title, to save typing and for consistency:
\def\Rutherford{Rutherford School Physics}
\def\Concepttitle#1{\noindent\textsc{\Rutherford\vspace{0.4cm}\\ \LARGE Physical Concept: \textbf{#1}}}
\def\Problemtitle#1{\noindent\textsc{\Rutherford\vspace{0.4cm}\\ \LARGE Website Problems: \textbf{#1}}}
\def\AddProblemtitle#1{\noindent\textsc{\Large \Rutherford ~ --- ~ Additional Problems\vspace{0.4cm}\\ \LARGE \textbf{#1}}}
%\def\AddProblemtitle#1{\noindent\textsc{\Rutherford\vspace{0.4cm}\\ \LARGE Additional Problems: \textbf{#1}}}

%define quick question to be used in eg concept sheet.
%\def\qq#2{#1}{\color{red}[#2]\color{black}}
\newcommand{\qq}[2]{\nl Quick Question:\hspace{1 mm} #1\color{red}\hspace{2 mm} Answer:\color{black}\hspace{1 mm}  #2}
\newcommand{\stress}[1]{\emph{#1}}

%%%%%%%%%%%   some definitions used in latexing the CQMP:
% fractions that are of right size in set equations
\def\half{{\textstyle \frac{1}{2}}}
\def\quarter{{\textstyle \frac{1}{4}}}
\def\third{{\textstyle \frac{1}{3}}}
\def\eighth{{\textstyle \frac{1}{8}}}

% obtain a new line
\def\nl{\hfil\break}
\def\nll{\\ \\ \noindent}



%creates numbered lists with a), then i.
\renewcommand{\theenumi}{\alph{enumi}}% first level are latin characters
\renewcommand{\labelenumi}{\theenumi)} %tells it to put a bracket after the character.
\renewcommand{\theenumii}{\roman{enumii}}%second level are little roman characters
\renewcommand{\labelenumii}{\theenumii.} %tells it to put a dot after the character
 % In a file called "Definitions.tex" in the same directory as this file.

%Define some boolean switches:
\newtoggle{solutions_only}	%Print only the solutions
\newtoggle{no_solutions}		%Don't print any solutions  (overridden by solutions_only)
\newtoggle{solutions_at_end}	%Print the solutions at end (overridden by solutions_only and no_solutions)
\newtoggle{no_credits}		%Don't print the credit arguments

%Use this to write a list of things needed to know for a section. It automatically won't print when "solutions_only" is on.
%Its only argument should be a list of things needed to know in "\item [....]" form
\newenvironment{knowledge}[1]{
\iftoggle{solutions_only}{}{It is assumed that students will be familiar with the following concepts:
\begin{itemize} #1 \end{itemize}
\vspace{0.5cm}}
}

%Allows the headings to be managed when not printing problems ect.
\newenvironment{Qsection}[1]{
%\iftoggle{solutions_only}{}{\section{#1}} %Don't output headings in the solutions(?)
\iftoggle{solutions_at_end}{\AtEndDocument{\section{#1}}}{}
\section{#1}
}

\newenvironment{Qsubsection}[1]{
%\iftoggle{solutions_only}{}{\subsection{#1}} %Don't output headings in the solutions(?)
\iftoggle{solutions_at_end}{\AtEndDocument{\subsection{#1}}}{}
\subsection{#1}
}

%Set the values of the boolean switches: Yes - "toggletrue", No - "togglefalse".
\togglefalse{solutions_only}	%	ONLY		Output only solutions? 
\togglefalse{no_solutions}		%	NONE		Don't output solutions at all? 
\togglefalse{solutions_at_end}	%	END		Output solutions at the end?
\togglefalse{no_credits}		%			Don't output the credit field
%All 8 cases have been tested; ONLY takes precedence, then NONE and finally END is lowest.


%##############################################################################################################
%											Then the bulk of the layout options:
%##############################################################################################################

\setlength{\topmargin}{-2cm}
%\setlength{\oddsidemargin}{0.5cm}
%\setlength{\evensidemargin}{0.5cm}


%##############################################################################################################


\newcounter{exercisenumber}%[chapter] %counter is set to zero when "chapter" appears
\def\theexercisenumber{\arabic{exercisenumber}}


\iftoggle{no_solutions}{}{ %Put a header at the end before the solutions, and reset the counter. Only if solutions are being printed AND at the end.
	\iftoggle{solutions_only}{}{
		\iftoggle{solutions_at_end}
			{\AtEndDocument{\newpage \part*{Solutions:} \setcounter{exercisenumber}{0} \setcounter{section}{0}}}{}
	}
}


%%%%%%%%%%%%%%%%%%%%%%%%%%%%%%%%%%%%%%%%%%%%%%%%%%%%%%%%%%%%%%%%%%%%%%%%%%%%%%%%%
%Creates \begin{problem}[label]{exercise_text}{source_text}{solution_text}\end{problem} command - the label argument is optional
%If put in, remember to put in [] brackets.  A label called label.ex will be generated.
\newenvironment{problem}[4][noref]{
 \refstepcounter{exercisenumber} %\refstepcounter allows you to reference to the exercise number
%
\iftoggle{solutions_only}{\hfil\break \textit{Solution}~\theexercisenumber:  #4}{ %If only solutions, just output solution.
	\noindent{\textit{Exercise}~\theexercisenumber:}
	\ifthenelse{\equal{#1}{noref}}{}{\label{#1.ex}} #2 %\vspace{0.3cm}
	\iftoggle{no_credits}{}{
			%\hfil\break {\small #3} \vspace{0.3cm} %This is the old line, replaced with the one below, without the ifthenelse statement; in case something goes wrong.
			\ifthenelse{\equal{#3}{}}{}{ {\tiny [#3]} \vspace{0.3cm}} %If the credit field is blank; don't bother printing it or the space for it.
	} %reference argument
%
	\iftoggle{no_solutions}{}{ %If the solutions aren't to be printed, do nothing.
		\iftoggle{solutions_at_end}
			{\AtEndDocument{\stepcounter{exercisenumber}\hfil\break \textit{Solution}~\theexercisenumber:  #4 \vspace{0.5cm}}} %If at the end: do this.
			{\hfil\break \textit{Solution}~\theexercisenumber:  #4} 	%Else leave in line as in TeX file.
	}
}

\vspace{0.2cm}}

%%%%%%%%%%%%%%%%%%%%%%%%%%%%%%%%%%%%%%%%%%%%%%%%%%%%%%%%%%%%%%%%%%%%%%%%%%%%%%%%%
%Creates \begin{hint}[label]{exercise_text}{hint_text}{source_text}{solution_text}\end{hint} command - the label argument is optional
%If put in, remember to put in [] brackets.  A label called label.ex will be generated.
\newenvironment{hint}[5][noref]{
 \refstepcounter{exercisenumber}
%
\iftoggle{solutions_only}{\hfil\break \textit{Solution}~\theexercisenumber:  #5}{  %If only solutions, just output solution.
	\noindent{\textit{Exercise}~\theexercisenumber:}
	\ifthenelse{\equal{#1}{noref}}{}{\label{#1.ex}} #2 \vspace{0.1cm}
	 \hfil\break  \textit{Hint:}  #3{} %\vspace{0.3cm}
	\iftoggle{no_credits}{}{
			\ifthenelse{\equal{#4}{}}{}{\\ \hfil {\tiny #4} \vspace{0.3cm}} %This is the old line, replaced with the one below, without the ifthenelse statement; in case something goes wrong.
			%\ifthenelse{\equal{#4}{}}{}{{\tiny [#4]} \vspace{0.3cm}} %If the credit field is blank; don't bother printing it or the space for it.
	} %reference argument
%
	\iftoggle{no_solutions}{}{%If the solutions aren't to be printed, do nothing.
		\iftoggle{solutions_at_end}
			{\AtEndDocument{\stepcounter{exercisenumber}\hfil\break \textit{Solution}~\theexercisenumber:  #5 \vspace{0.5cm}}} %If at the end: do this.
			{\hfil\break \textit{Solution}~\theexercisenumber:  #5}%Else leave in line as in teX file.
	}
}
\vspace{0.2cm}}

%%%%%%%%%%%%%%%%%%%%%%%%%%%%%%%%%%%%%%%%%%%%%%%%%%%%%%%%%%%%%%%%%%%%%%%%%%%%%%%%%


 %%%%%%%
\newenvironment{additional}[2][noref]{
 \refstepcounter{exercisenumber}
% \vspace{.2cm}
\nl
\noindent{\textit{Exercise}~\theexercisenumber:}
\ifthenelse{\equal{#1}{noref}}{}{\label{#1.ex}} #2 }{%\vspace{5.1cm}
 }
%










% to mark as a draft.  Comment both these lines out when complete.  The page number will return to the footer.
%\pagestyle{myheadings}
%\markright{\textcolor{red}{\textbf{DRAFT: \today}}}


\begin{document}
\addtolength{\topmargin}{-0.7 cm}
\setlength{\columnsep}{22pt}
\Concepttitle{Energy, work and power}

\section{Definitions}
Energy, either potential or kinetic (due to motion), is possessed by a body.  It can be converted into other forms and will ultimately degrade to heat energy (the random motions of atoms and molecules).  The unit of energy is the Joule.
\subsection{Potential energy:}Bodies possess potential energy by virtue of their position in space, state of charge, . . .\\
For instance:
\begin{enumerate}
  \item \textit{Gravitational potential energy} is:
\begin{equation*} 
    V = m g h \;\;\; \textrm{or} \;\;\; V = - m_1m_2 G/r.
\end{equation*}
The former is for a body of mass $m$ in a uniform gravitational field with acceleration $g$ when it is a height $h$ above a reference level.  For instance, this formulation is suitable for bodies close to the surface of the Earth where the gravitational field (pull from the Earth) does not vary much with height.\\
The latter is for or two masses $m_1$ and $m_2$ separated by a distance $r$, with $G$ the universal gravitational constant.  The negative sign indicates attraction -- the energy is reduced further as the masses approach each other.  See below for the connection between force, work and potential energy change.
  \item \textit{Electric potential energy} is:
   \begin{equation*} 
   V = - q E h \;\;\; \textrm{or} \;\;\; V = q_1q_2/(4\pi \epsilon_0 r).
\end{equation*} 
The former is for a body of charge $q$ in a uniform electric field $E$ with $h$ the distance moved by the charge along the electric field from a reference position.\\
In general fields vary with position, and the latter is the potential energy associated with two charges $q_1$ and $q_2$ separated by $r$, where $\epsilon_0$ is a constant known as the permittivity of free space.  The potential energy is positive if $q_1$ and $q_2$ are both of the same sign -- like charges repel and lower their energy by moving apart.
\end{enumerate}
\subsection{Kinetic energy:}  Bodies possess energy known as kinetic energy, $T$, by virtue of their motion.  Translational motion with speed $v$, or rotational motion with angular speed $\omega$, lead to kinetic energies:
 \begin{equation*} T = \half m v^2 \;\;\; \textrm{or} \;\;\; T = \half I \omega^2.
\end{equation*}
Here $m$ is the particle mass, and $\half m v^2$ is the familiar kinetic energy.  The angular equivalent of mass is the moment of inertia $I$  --- see notes on angular motion.  Kinetic energy takes many forms.  We focus on that due to simple translational motion.

\section{Conservation of energy}
In the absence of friction (resisting motion and leading to heat), the sum $E$ of the kinetic and potential energy is conserved.
\begin{equation*} E = T + V =  \textrm{constant}.
\end{equation*}
The energy can be converted between the forms $T$ and $V$ by the particle doing work, or work being done on the particle. So when $V(x)$ varies with position $x$, then so will $T = E - V(x)$ also vary.  In turn, speed $v(x)$ must vary with position since $T = \half m v^2$.  This approach to dynamics is explored in ``Conservation Methods in Dynamics".\\ When there is friction, then heat $Q$ can be produced from the motion.  A stricter statement of energy conservation is
\begin{equation*} E = T + V + Q =  \textrm{constant}.
\end{equation*}
However, conversion of energy  back from heat $Q$ to kinetic $T$ or potential energy $V$ is never perfect and leads us to the study of Thermodynamics.
\subsection*{Questions}
\qq{What is the gravitational potential energy of a mass of 1kg resting 1m above the ground? [Take $g= 10\textrm{ms}^{-2}$.]  After being dropped, what speed does the mass attain before hitting the ground?  How much heat is produced when motion has ceased?}{$E=10$ J, $v=2\sqrt{5}$ ms$^{-1}$, $Q=10$ J.}
\qq{By considering energy, show that the speed $v$ reached when a particle is dropped from rest through a height $h$ is $v^2 = 2 g h$.  Show that if it initially has a downward speed $u$ as it starts dropping through $h$, then its final speed is $v^2 = u^2 + 2 gh$.  [Notice how this energy method yields the result ``$v^2 - u^2 = 2 as$" familiar from kinematics.]}{There is no friction, therefore the initial energy will equal the final energy. Thus, $m g h = 1/2 m v^2$, which gives: $v^2 = 2 g h$. If the particle has an initial velocity, then $m g h + \frac{1}{2} m u^2 = \frac{1}{2} m v^2$, instead.}
\qq{A particle of mass $m$ approaches a slope with initial speed $v_0$.  It moves up the slope conserving energy.\\
 (i)  What is its initial energy?\\
(ii) To what vertical height $h_f$ does it rise when it finally stops?\\
(iii) What speed is it travelling at when it attains heights $h_f/4$, $h_f/2$ and $3h_f/4$?  What is the speed at these heights on the way down when the particle reverses direction at the top of its trajectory?}{(i)$E_i=\frac{1}{2} m v^2$; (ii) $h_f=\frac{1}{2} v^2/g$; (iii) $h_f/4$: $u =\frac{\sqrt{3}v}{2}$, $h_f/2$: $u = \frac{u}{\sqrt{2}}$, $3h_f/4$: $u = \frac{v}{4}$}

  \subsection{Work}
  Work $W$ is given by
\begin{equation*} 
W = {F}.{x}
\end{equation*}
where ${x}$ is the distance moved by the point of application of the force ${F}$ in the direction of the force.  The units of work are J = N . m. Evidently there is no work done if ${F}$ is perpendicular to ${x}$ --- we more properly refer to force and displacement as vectors, ${\vtr F}$ and ${\vtr x}$. \nll
{\bf Examples are:}\\
    (a) picking up a mass with a vertical force and raising it through a height $h$.  In that case $f = mg$ and $x = h$ whereupon $W = mg h$.  Note that the work done is stored as gravitational potential energy of the form discussed above.  Clearly forces arise when potentials vary with position.  \\ If the mass is allowed to drop back, the gravitational force can be used to e.g. turn an electrical generator and electrical work can be recovered.  That in turn could be used to charge a battery and the energy would finally be chemical potential energy.\nll
    (b) circular motion of a mass at the end of a string with the tension in the string creating the centripetal acceleration.  The force and acceleration are perpendicular to the motion and no work is done.  The energy (and speed) of the particle do not increase as the motion elapses.

\subsection*{Questions}
\qq{A horizontal force $f$ pulls a mass on a rough surface of sliding friction $\mu_k$ through a distance $d$.  Show that the work done is $W = mg \mu_k d$.  Where does the work end up? [If necessary, consult the concept sheet on Friction.]}{As heat}\\
\qq{The same force is applied at an angle $\theta$ to the horizontal.  Show that now, $W =\mu_k\cos\theta(mg - f \sin\theta)d$.}{}
\end{document}
