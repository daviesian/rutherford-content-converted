%% ID: bolt_thrower
%% TITLE: Bolt Thrower
%% TYPE: question
%% QUESTIONTYPE: numerical
%% CONCEPTS: eq_of_motions_diff, forces, trig, vectors1
%% VIDEOS: 
%% LEVEL: 3
%% TOPIC: mechanics/kinematics
%% ORDER: 9

\begin{problem}[Tristan_Bolt_Thrower] %[A1989PIQ3i] %projectiles, SUVAT %adapted so loosely as to be unrecognisable
{\exposition{A castle wall has bolt throwers which fire a bolt horizontally at a speed \vari{v}. In order to fire over the enemy's shields, the bolt must make an angle of \quantity{\theta}{45\sup{\circ}} to the horizontal when it hits the ground. The bolt throwers can be mounted at different heights in the wall and set to fire at different speeds.}

\question{Find the maximum range $s$ at a given height $h$ and the speed at which the bolt thrower must fire in order to achieve this maximum range.}
 }
{\textit{Written by TR for The Rutherford Schools Physics Project.}}
{
\begin{figure}[h]
	\centering
	\includegraphics[width=0.4\textwidth]{Kinematics_bolt_projectile}
	\caption{The motion of the bolt with all relevant quantities defined.}
	\label{fig:Kinematics_projectile_arrow}
\end{figure}
\begin{figure}[h]
	\centering
	\includegraphics[width=0.3\textwidth]{Kinematics_bolt_velocity}
	\caption{The components of the bolt's velocity as the bolt hits the ground.}
	\label{fig:Kinematics_bolt_velocities}
\end{figure}
\\
\\
	We know that the bolt has to land at \quantity{\theta}{45\sup{\circ}} to the horizontal. As no force acts horizontally on the bolt, we can relate the final vertical component of velocity, \vari{v_y} to the speed at which the bolt is fired using trigonometry,
	\begin{equation*}\frac{v_y}{v}=\tan(\theta)\end{equation*}
	We know from SUVAT that the final velocity of a body initially at rest is equal to $at$ where $a$ is the acceleration. Therefore, the vertical velocity when the bolt hits the ground some time $t$ after it is fired is given by
	\begin{equation*}v_y=gt\end{equation*}
	It follows then that
	\begin{equation}\label{Kinematics_bolt_1}v=\frac{gt}{\tan(\theta)}\end{equation}
	This depends on \vari{t} which we do not want so, using SUVAT to find the time taken for the bolt to fall a height \vari{h},
	\begin{equation*}-h=-\frac{1}{2}gt^2\end{equation*}
	\begin{equation*}t^2=\frac{2h}{g}\end{equation*}
	\begin{equation}\label{Kinematics_bolt_2}t=\sqrt{\frac{2h}{g}}\end{equation}
	Substituting equation \ref{Kinematics_bolt_2} into equation \ref{Kinematics_bolt_1}
	\begin{equation*}v=\frac{g\sqrt{\frac{2h}{g}}}{\tan(\theta)}=\frac{\sqrt{2gh}}{\tan(\theta)}\end{equation*}
	Substituting in the value \quantity{\theta}{45\sup{\circ}},
	\begin{equation*}v=\sqrt{2gh}\end{equation*}
	The horizontal distance travelled when the bolt hits the ground is  the horizonal distance travelled moving at a speed \vari{v} for time \vari{t}. Hence using SUVAT:
	\begin{equation*}s=\sqrt{2gh}\cdot\sqrt{\frac{2h}{g}}\end{equation*}
	\begin{equation*}s=2h\end{equation*}
	\answer{The bolt thrower must throw at a speed \valuedef{v}{\sqrt{2gh}}{} to achieve the maximum range of \valuedef{s}{2h}{}.}
}
\end{problem}