%% ID: rocket_orbit
%% TITLE: Rocket in orbit
%% TYPE: question
%% QUESTIONTYPE: numeric
%% CONCEPTS: newtonii
%% VIDEOS: 
%% LEVEL: 1
%% TOPIC: mechanics/dynamics
%% ORDER: 8

\begin{problem}[Robin8]
{A rocket similar to that which launches NASA’s space shuttle into a low earth orbit has a take-off mass of \quantity{2000}{tonnes} (\quantity{1}{tonne}$ =  1\times$ \quantity{10^3}{kg}) and provides a thrust at take-off of \quantity{24}{MN} (\quantity{1}{MN}$ = 1\times$\quantity{10^6}{N}). The thrust remains constant and \valuedef{g}{10}{m\,s\sup{-2}}.  Air resistance can be neglected at take-off. 
\begin{enumerate}
	\item Sketch a diagram of the rocket and mark on the forces acting during take-off.
	\item Calculate the acceleration of the rocket at take-off.
	\item What fraction of the acceleration due to gravity (\vari{g}) is this?
	\item To reach orbit, the shuttle has to achieve a speed of \quantity{8}{km\,s\sup{-1}}.  How long would this take with this acceleration?
	\item The fuel is burnt off at the rate of \quantity{4.0}{tonnes} each second.  For how long can the engines provide the thrust?
	\item What simple explanation is there for the large difference in the two times that you have calculated in (d) and (e) for the shuttle to reach orbit?
\end{enumerate}
}
{\stress{Question by RWH}}
{}
\end{problem}