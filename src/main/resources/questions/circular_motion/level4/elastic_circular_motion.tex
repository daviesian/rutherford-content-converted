%% ID: elastic_circular_motion
%% TITLE: Elastic Circular Motion
%% TYPE: question
%% QUESTIONTYPE: scq
%% CONCEPTS: angular_circular, resolving_vectors
%% VIDEOS: A1987PIQ9l.mp4
%% LEVEL: 4
%% TOPIC: mechanics/circular
%% ORDER: 4

\begin{problem}[A1987PIQ9l] 
{\exposition{A mass, \vari{m} of \quantity{0.050}{kg} is attached to one end of a piece of elastic of unstretched length, \valuedef{l}{0.50}{m}. The force constant, \vari{k} of the elastic (i.e. the force required to produce unit extension) is \quantity{40}{N\,m\sup{-1}}. The mass is rotated steadily on a smooth table around a fixed point in a horizontal circle of radius \valuedef{r}{0.70}{m}.}
\question{What is the approximate speed of the mass?}
\begin{enumerate}
	\item \choice[a]{\quantity{11}{m\,s\sup{-1}}}\correct
	\item \choice[b]{\quantity{15}{m\,s\sup{-1}}} 
	\item \choice[c]{\quantity{20}{m\,s\sup{-1}}}
	\item \choice[d]{\quantity{24}{m\,s\sup{-1}}}
	\item \choice[e]{\quantity{28}{m\,s\sup{-1}}}
\end{enumerate}}
{\stress{Adapted with permission from UCLES, A Level Physics, June 1987, Paper 1, Question 9.}}
{\answer{The correct answer is (a).} The tension from the extension, \valuedef{T}{kx}{} can be equated to the force from the radial acceleration, \valuedef{a_{R}}{m\frac{v^{2}}{r}}{}, as these are the only two forces acting radially during the motion of \vari{m}. The velocity can be calculated from: \valuedef{\frac{mv^{2}}{r}}{kx}{}. Since the extension, \vari{x} is \quantity{(0.70 - 0.50)}{m} = \quantity{0.20}{m}, \valuedef{v}{\sqrt{\frac{kxr}{m}}}{} = \quantity{\sqrt{\frac{(40)(0.2)(0.7)}{0.05}}}{m\,s\sup{-1} = \quantity{10.58}{m\,s\sup{-1}} \approx \quantity{11}{m\,s\sup{-1}}.}
\end{problem}