%% ID: angle_between_identical_masses
%% TITLE: Angle Between Two Identical Masses
%% TYPE: question
%% QUESTIONTYPE: scq
%% CONCEPTS:  energy, momentum, zero_momentum_frame, vectors
%% VIDEOS: 
%% LEVEL: 6
%% TOPIC: mechanics/dynamics
%% ORDER: 1

\begin{problem}[angle_between_identical_masses] 
{A particle with mass \vari{m} travelling at speed  \vari{u} collides with a stationary particle of the same mass. What is the angle between the velocities of the two masses after this collision?
\begin{enumerate}
	\item $26.6^{\circ}$
	\item $45^{\circ}$
	\item $53.2^{\circ}$
	\item $90^{\circ}$ \answer
	\item $180^{\circ}$
\end{enumerate}
}
{\textit{Another old question from multiple places}}
{The correct answer is (d). By conservation of energy, $\half mu^2 = \half mv_1^2 + \half mv_2^2$, where \vari{v_1} and \vari{v_2} are the speeds of the particles after the collision. In the zero momentum frame, both particles have a speed of \vari{\frac{u}{2}} in opposite directions, so on converting back into the lab frame, a triangle is created with sides of length  \vari{u},  \vari{v_1} and  \vari{v_2}, with an angle \vari{\theta} between the final velocities of the two particles. Using the cosine rule, $u^2 = v_1^2 +v_2^2 - 2 v_1 v_2 \cos \theta$. Combining this equation with that for conservation of energy gives $2 v_1 v_2 \cos \theta = 0$, which is only possible if $\cos \theta = 0$, so $\theta = 90^{\circ}$

\begin{figure}[h]
	\centering
	\includegraphics[width=0.5\textwidth]{Dynamics_angle_between_identical_masses}
	\caption{}
	\label{fig:Dynamics_angle_between_identical_masses}
\end{figure}


}
\end{problem}