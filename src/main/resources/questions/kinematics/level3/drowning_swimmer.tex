%% ID: drowning_swimmer
%% TITLE: Saving a drowning swimmer 
%% TYPE: question
%% QUESTIONTYPE: numerical
%% CONCEPTS: eq_of_motions_diff
%% VIDEOS: 
%% LEVEL: 3
%% TOPIC: mechanics/kinematics
%% ORDER: 10


\begin{problem}[Jan_Drowning_Sea] 
{\exposition{A swimmer is drowning at B, a distance \vari{a/2} away from the beach as in Figure \ref{fig:Kinematics_Drowning_Sea_1}. A lifeguard patrolling along the very edge of the beach spots the swimmer when the lifeguard is at point A, which a distance\vari{a} away from the point on the beach that is closest to the drowning swimmer. The average speed of the lifeguard running along the beach is \vari{v}, and her average swimming speed is\valuedef{v_s}{v/3}{}.
\begin{figure}[h]
	\centering
	\includegraphics[width=0.31\textwidth]{Kinematics_Drowning_Sea_1}
	\caption{}
	\label{fig:Kinematics_Drowning_Sea_1}
\end{figure}}

	\begin{enumerate}
	\item \question[a]{How long would it take for the lifeguard to reach the drowning swimmer at B if she entered the water at A and started swimming immediately?}
	\item \question[b]{Instead she runs along the beach to a point C before she starts swimming. Find the quickest time, \vari{t_{min}}, in which the lifeguard can reach the drowning swimmer.}
	\end{enumerate}

\exposition{Leave your answers in terms of $a$ and $v$ only.}}
\hinta{For part (b), let the distance AC be \vari{x}. A useful result for when $f(x) = \sqrt{x^2 + (a)^2}$ is $f'(x) = \frac{x}{\sqrt{x^2 + (a)^2}}$.}
{\textit{Written by JZ and NHB for The Rutherford Schools Physics Project.}}
{	\begin{enumerate}
	\item This is quite a simple application of Pythagoras' Theorem. The lifeguard will swim the distance AB at a speed \valuedef{v_s}{v/3}{}, hence we can find the time taken, \vari{t}, if we know distance AB:
	
\begin{align*} AB = \sqrt{a^2+\left(\frac{a}{2}\right)^2} = \sqrt{\frac{5}{4}a^2} = \frac{\sqrt{5}}{2}a \end{align*}

\begin{align*} \textrm{$t =$ distance/speed } = \frac{\left(\frac{\sqrt{5}}{2}a\right)}{\left(\frac{v}{3}\right)} = \frac{3\sqrt{5}a}{2v} \end{align*}

\answer[a]{The lifeguard will take a time \valuedef{t}{\frac{3\sqrt{5}a}{2v}}{}}

	\item Now we need to consider two parts of the journey. See Figure \ref{fig:Kinematics_Drowning_Sea_2}.
	
\begin{figure}[h]
	\centering
	\includegraphics[width=0.3\textwidth]{Kinematics_Drowning_Sea_2}
	\caption{}
	\label{fig:Kinematics_Drowning_Sea_2}
\end{figure}

The time taken, \vari{t}, is given by:

\begin{align*} t &= \frac{\textrm{AC}}{v} + \frac{\textrm{CB}}{\left(\frac{v}{3}\right)}\\
&= \frac{a-x}{v} + \frac{3\sqrt{x^2 +\left(\frac{a}{2}\right)^2}}{v}
\intertext{For minimum:} \frac{\textrm{d}t}{\textrm{d}x} &= \frac{-1}{v} + \frac{3x}{v\sqrt{x^2 +\left(\frac{a}{2}\right)^2}} = 0\\
3x &= \sqrt{x^2 + \left(\frac{a}{2}\right)^2} \\
x^2 &= \frac{a^2}{4 \times 8} \\
x &= \frac{\sqrt{2}a}{8}\end{align*}

Now we just put this value of $x$ back into our equation for time, and this should be a minimum value:

\begin{align*} t_{min} &= \frac{a-\frac{\sqrt{2}a}{8}}{v} + \frac{3\sqrt{\frac{a^2}{32} +\left(\frac{a}{2}\right)^2}}{v}\\ \\ &= \frac{(8-\sqrt{2})a + 9\sqrt{2}a}{8v} \\
\\ &= (1+\sqrt{2})\frac{a}{v} \end{align*}

\answer[b]{The lifeguard will now take a time \valuedef{t_min}{ (1+\sqrt{2})\frac{a}{v}}{}

\end{enumerate}

}
\end{hint}