% This is the template that sets out all of the Problems and produces the Exercise/Solution labels and numbering
% There are two classes of Exercise: "problem" which has a Question and Solution, and "hint" which has a Question, Hint and Solution

% These are the packages to use in all documents, and the paper size to use:
\documentclass[a4paper,11pt]{article}
\usepackage[usenames,dvipsnames]{xcolor}
\usepackage[margin=1.5cm]{geometry}
\usepackage{amsmath}
\usepackage{amssymb}
\usepackage{color}
\usepackage{graphicx}
\usepackage{graphics}
\usepackage[margin=1.5cm]{geometry}
\usepackage{fancyhdr}
\usepackage{float}
\usepackage{lscape}
\usepackage[font={small},labelfont=bf]{caption}
\usepackage{ifthen}
\usepackage{enumitem}
\usepackage{subcaption}		%Allows grouped figures. The percentage sign after the first \end{subfigure} puts them side by side, omitting it puts one above the other.
\usepackage{graphicx,xcolor} 	%Allows the use of colour in the files
\usepackage{centernot} 		%Puts the / in a not equal to sign in the centre, use as \cnot{...}
\usepackage{comment} 		%Allows \begin{comment} .... \end{comment} to comment out bulk text.
\usepackage{etoolbox}		%Allows the boolean flags and the \toggletrue and \togglefalse commands
\usepackage{cancel}		%Allows the crossing out of terms in maths mode to show they cancel out
\usepackage{wrapfig}

%Packages for font choices
\usepackage{palatino}
\usepackage{mathpazo}