%% ID: three_spheres
%% TITLE: Three spheres
%% TYPE: question
%% QUESTIONTYPE:  numerical
%% CONCEPTS: forces, moments, newtoni
%% VIDEOS: 
%% LEVEL: 5
%% TOPIC: mechanics/statics
%% ORDER: 10

%Solution by Jan Zamirski
\begin{problem}[MW_Spheres_Shortened]
{\question{Three spheres are in a triangular arrangement on a plane, spot welded together where they touch. Another sphere is placed on top of the first three. Take the spheres to be identical, of mass \vari{m} and radius \vari{a}. Assume where necessary that the contacts are smooth.  What is the tension in the weld necessary to keep the pyramid stable when the mass is 0.8 kg?}
}
{\textit{Created for the Rutherford School Physics Project by MW and JZ.}}
{\answer{\valuedef{T}{\frac{mg}{3\sqrt{6}}}{.}  Substituting in \valuedef{m}{0.8}{kg}, we find \valuedef{T}{1.1}{N}}
The first thing to do is to draw a (side-on) diagram, shown on the left in Figure \ref{fig:Statics_Spheres_1}, which considers the tension in the weld, $T$, (remember there are two welds on each sphere) as well as the reaction force, $R$, between the top sphere and each contact (there is one contact between each welded ball and the top sphere, for a total of three). The only weight we need to consider is that of the top sphere, which exerts a downwards force $mg$.

\begin{figure}[h]
	\centering
	\includegraphics[width=0.7\textwidth]{../../../figures/Statics_Spheres_1.svg}
	\caption{}	
	\label{fig:Statics_Spheres_1}
\end{figure}

We can immediately see that resolving vertically for the top sphere will give us $R$, in turn allowing us to resolve horizontally for the bottom spheres for $T$. The angle we need for this is $\theta$, whose position is displayed on the right of Figure \ref{fig:Statics_Spheres_1}. We can find $\theta$ in terms of $a$ by using the known angles from the equilateral triangle (all are $60 ^{\circ}$, and in this case we want half that angle) and applying trigonometry. See Figure \ref{fig:Statics_Spheres_2}. 

\begin{figure}[h]
	\centering
	\includegraphics[width=0.55\textwidth]{Statics_Spheres_2}
	\caption{}	
	\label{fig:Statics_Spheres_2}
\end{figure}

\begin{align*} &2a\cos{(\theta)} = b \\
\intertext{Where:}& b = \frac{a}{\cos{(30)}} \\
\intertext{So:} &\cos{(\theta)}= \frac{1}{2\cos{(30)}} = \frac{\sqrt{3}}{3}\\ \end{align*}

Now resolve vertically for the top sphere (see Figure \ref{fig:Statics_Spheres_1}):

\begin{align*} &3R\sin{(\theta)} = mg \\
&R = \frac{mg}{3\sin{(\theta)}} \end{align*}

Finally, resolve horizontally for the bottom sphere (Figure \ref{fig:Statics_Spheres_4}):

\begin{figure}[h]
	\centering
	\includegraphics[width=0.2\textwidth]{Statics_Spheres_4}
	\caption{}	
	\label{fig:Statics_Spheres_4}
\end{figure}

\begin{align*} &2T\cos{(30)} = R\cos{(\theta)} \\
\intertext{substituting for $R$, and using a triangle to work out $\sin{(\arccos{x})}$ gives us the necessary tension in the weld, $T$:} &T = \frac{mg\cos{(\theta)}}{(2\cos{(30)})(3\sin{(\theta)})} \\
\\ &T = \frac{mg\left(\frac{\sqrt{3}}{3}\right)}{6\cos{(30)}\sin{\left(\arccos{\left(\frac{\sqrt{3}}{3}\right)}\right)}} \\
\\ &T = \frac{mg}{3\sqrt{6}}\end{align*}

Substituting in $m$ = 0.8 kg, find $T$ = 1.1 N
}
\end{problem}