%% ID: swinging_rod
%% TITLE: A swinging rod
%% TYPE: question
%% QUESTIONTYPE: numerical
%% CONCEPTS: shm
%% VIDEOS: 
%% LEVEL: 4
%% TOPIC: mechanics/shm
%% ORDER: 6


%Physics: period, SHM, omega
%Maths: trig, trig calculus
\begin{problem}[IntA1987APIQ1p]
{\exposition{A uniform rod of mass \valuedef{m}{200}{kg} is supported horizontally by two vertical ropes of length \valuedef{L}{14.0}{m} which are attached to each end of the rod. It is made to swing like a simple pendulum with oscillations of amplitude \valuedef{A}{0.50}{m}.}
\begin{enumerate}
		\item \question[a]{Calculate the period and the maximum acceleration of the rod in these oscillations.}
		\item \question[b]{What would the period of the oscillations be if the length of the supporting ropes were reduced to \quantity{9.0}{m}?}
		\item \question[c]{What would the period be if a rod with only half the mass of the original one were suspended from the ropes of length \quantity{14.0}{m}?}
\end{enumerate}
}
{\textit{Adapted with permission from UCLES, A Level Additional Physics, November~1987, Paper~1, Question~1.}}
{This question can actually be attempted without too much mathematics, and provided we use the assertion that is oscillates like a simple pendulum:
\begin{enumerate}
	\item If the rod is undergoing SHM with a small amplitude, we can approximate it as just moving horizontally in order to find the acceleration.
If $x = A\cos(\omega t)$, where we will simply quote the angular frequency of a pendulum $\omega^{2} = \frac{g}{l}$, then the acceleration is $\ddot{x} = -\omega^{2}A \cos(\omega t)$ which has a maximum value of:
\begin{align*} 
a_{\textrm{max}} &= \omega^{2}A \\ 
&= \left(\frac{g}{l}\right)A \\ 
&= \frac{(9.8)(0.5)}{(14)} \textrm{ ms}^{-2} \\ 
&= 0.35 \textrm{ ms}^{-2} 
\end{align*}

\answer[a]{The period of oscillations would be \valuedef{T}{7.5}{s} and the maximum acceleration of the rod would be \valuedef{a}{0.35}{m s\sup{-2}}.}

	\item The equation linking period to the length of the pendulum is: 
\begin{equation*}
 T = \frac{2 \pi}{\omega} = 2\pi \sqrt{\frac{l}{g}}
\end{equation*} 
and so putting in the new length of the pendulum, we find: 
\begin{align*}
T_{\textrm{new}} &= 2\pi \sqrt{\frac{l_{\textrm{new}}}{g}} \\ 
&= 2\pi \sqrt{\frac{(9)}{(9.8)}} \textrm{ s} \\ 
&= 6.02 \textrm{ s}

\answer[b]{The period of oscillations would be \valuedef{T}{6.0}{s}}

\end{align*}
	\item \answer[c]{The motion of a simple pendulum is independent of mass, and so the period will be the same as for the original rod, \valuedef{T}{7.5}{s}.}
\end{enumerate}
}
\end{problem}