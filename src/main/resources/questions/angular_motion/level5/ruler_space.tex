%% ID: ruler_space
%% TITLE: Ruler in space
%% TYPE: question
%% QUESTIONTYPE: scq
%% CONCEPTS: momentum, impulse, circ_mot, ang_vel, ang_mom
%% VIDEOS: 
%% LEVEL: 5
%% TOPIC: mechanics/dynamics
%% ORDER: 2

\begin{problem}[Tripos_Ruler_Length]
{\exposition{A an astronaut in a shuttle orbiting Earth notices a ruler floating in the capsule. She gives one end of it a sharp tap in a direction perpendicular to its length. Consequently its centre of mass moves away at a speed of \quantity{0.4}{m\,s\sup{-1}} and it performs a full rotation every \quantity{0.8}{s}.}
\question{What is the length of the ruler?}
\begin{enumerate}
\item choice[a]{\quantity{0.05}{m\,s\sup{-1}}}
\item choice[b]{\quantity{0.08}{m\,s\sup{-1}}}
\item choice[c]{\quantity{0.15}{m\,s\sup{-1}}}
\item\choice[d]{\quantity{0.31}{m\,s\sup{-1}}}\correct
\item choice[e]{\quantity{2.47}{m\,s\sup{-1}}}
\end{enumerate}
}
{%\textit{Used/Adapted with permission from UCLES, A/O Level Physics/Maths/Physical Science, June/November 198, Paper, Question.}
}
{\answer{\valuedef{l}{0.31}{m}}
Figure \ref{fig:AngularMotion_astro} shows the impulse $\Delta p$ being applied at one end of the rod:
\begin{figure}[h]
\centering
\includegraphics[width=7cm]{AngularMotion_astro}
\caption{}
\label{fig:AngularMotion_astro}
\end{figure}
\\
Since the rod is stationary before the tap, all of its linear and angular momentum afterwards will be due to the impulse of magnitude $\Delta p$. So 
\begin{equation*}
\Delta p=mv
\end{equation*}
and
\begin{equation*}
\frac{l}{2}\Delta p=I\omega=\frac{1}{12}ml^2\omega
\end{equation*}
for centre of mass velocity $v$, angular speed $\omega$. Substituting for $\Delta p$ in the second equation we find
\begin{align*}
\frac{l}{2}mv=\frac{1}{12}ml^2\omega \\
\Rightarrow l=\frac{6v}{\omega}
\end{align*}
Using $\omega=2\pi/T$
\begin{align*}
l=\frac{3vT}{\pi}=0.3\textrm{ m}
\end{align*}
}
\end{problem}