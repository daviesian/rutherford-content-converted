%% ID: highway_pursuit
%% TITLE: Highway Pursuit
%% TYPE: question
%% QUESTIONTYPE: scq
%% CONCEPTS: eq_of_motions_diff
%% VIDEOS: 
%% LEVEL: 3
%% TOPIC: mechanics/kinematics
%% ORDER: 1

\begin{problem}[Jan_Highway_Pursuit] 
{\exposition{In a high speed highway pursuit, the police are chasing some gangsters. The police car is travelling at \quantity{50}{ ms\sup{-1}} and the gangsters are travelling at \quantity{52}{ ms\sup{-1}}.} \question{By considering the velocity of the gangsters relative to the police, what is the distance between the two cars after 90 seconds? Assume they start at the same point (the crime scene).}
\begin{enumerate}
	\item \choice[a]{90 m}
	\item \choice[b]{4500 m}
	\item \choice[c]{180 m}\correct
	\item \choice[d]{4680 m}
	\item \choice[e]{2 m}
\end{enumerate}
}
{\textit{Written by JZ for The Rutherford Schools Physics Project.}}
{\answer{The answer is (c).} Since the cars are travelling on the same line (in the same direction), the velocity of the gangsters relative to the police is given by ``gangsters relative to the road'' minus ``police relative to the road'', which is $52 - 50 = 2\textrm{ ms}^{-1}$. Then use distance = speed $\times$ time to find the distance, \vari{s}, between the two cars: $s = vt = 2 \times 90 = 180 \textrm{m}$. Note that \quantity{4500}{m} and  \quantity{4680}{m} are the distances travelled by the police and gangsters (relative to the road) respectively in this time; however we are not interested in this, as the gangsters are not escaping from a stationary point.
}
\end{problem}