%% ID: rapid_acceleration
%% TITLE: Rapid Acceleration
%% TYPE: question
%% QUESTIONTYPE: scq
%% CONCEPTS: newtonii, hooke
%% VIDEOS: 
%% LEVEL: 5
%% TOPIC: mechanics/dynamics
%% ORDER: 1

\begin{problem}[IntA1986PIQ4] 
{\exposition{A car of mass \vari{m} has an engine which can deliver power \vari{P}.} \question{What is the minimum time in which the car can be accelerated from rest to speed \vari{v}?}
\begin{enumerate}
	\item \choice[a]{\vari{\frac{mv}{P}}}
	\item \choice[b]{\vari{\frac{P}{mv}}}
	\item \choice[c]{\vari{\frac{mv^{2}}{2P}}}\correct
	\item \choice[d]{\vari{\frac{2P}{mv^{2}}}}
	\item \choice[e]{\vari{\frac{mv^{2}}{4P}}}
\end{enumerate}}
{\stress{Used with permission from UCLES, A Level Physics, November 1986, Paper 1, Question 4.}}
{\answer{The correct answer is (c). This can be seen since Energy Gained = Power $\times$ Time, and the energy we want to gain is the kinetic energy at speed \vari{v}, equal to \vari{\frac{1}{2}mv^{2}}. This gives \valuedef{Pt}{\frac{1}{2}mv^{2}}{}, which rearranges to the correct answer.}}
\end{problem}