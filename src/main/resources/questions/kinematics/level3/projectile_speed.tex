%% ID: projectile_speed
%% TITLE: Speed of a projectile
%% TYPE: question
%% QUESTIONTYPE: scq
%% CONCEPTS: eq_of_motions_diff, forces, newtonii, vectors1
%% VIDEOS: 
%% LEVEL: 3
%% TOPIC: mechanics/kinematics
%% ORDER: 4


\begin{problem}[A1988PIQ3l]
{\exposition{A ball is projected horizontally from the top of a cliff on the surface of the Earth with a speed of \quantity{40}{ ms\sup{-1}}.} \question{Assuming that there is no air resistance, what will its speed be \quantity{3}{s} later?}
\begin{enumerate}
	\item \choice[a]{30 ms$^{-1}$}
	\item  \choice[b]{40 ms$^{-1}$}
	\item  \choice[c]{50 ms$^{-1}$}\correct
	\item  \choice[d]{60 ms$^{-1}$}
	\item  \choice[e]{70 ms$^{-1}$}
\end{enumerate}
}
{\textit{Used with permission from UCLES, A Level Physics, June 1988, Paper 1, Question 3.}}
{\answer{The correct answer is (c).} The horizontal speed of the ball is unchanged as there is no resultant horizontal force acting on the ball, so $v_{x} = 40$ ms$^{-1}$. The vertical speed can be found using the SUVAT equation $v = u + at$ where \valuedef{v}{0}{}, \valuedef{a}{g}{} and \valuedef{t}{3}{s}, so $v_{y} = 9.8 \times 3 = 29.4$. The two components can then be put together using Pythagoras to obtain the overall speed $v =~\sqrt{v_{x}^{2} + v_{y}^{2}}~=~49.6\textrm{ ms}^{-1} \approx~50\textrm{ ms}^{-1}$.
}
\end{problem}