%% ID: record_ang_vel
%% TITLE: Old fashioned record
%% TYPE: question
%% QUESTIONTYPE: scq
%% CONCEPTS: moi, ang_mom, ang_vel
%% VIDEOS: 
%% LEVEL: 5
%% TOPIC: mechanics/dynamics
%% ORDER: 5

\begin{problem}[Ed_record_angular_velocity]
{\esposition{An old fashioned record (a disc) with moment of inertia \quantity{0.1}{kg\,m\sup2} is rotating freely about a vertical axis with an angular velocity of \quantity{1}{rad\,s\sup{-1}}. A pencil sharpener (much smaller than the disc so it can be treated as a point mass) of mass \quantity{10}{g} is placed a distance of \quantity{20}{cm} from the centre of the disc, without imparting any impulse.}
\question{What is the new angular velocity of the record?}
}
{\textit{Created for the Rutherford School Physics Project by ES.}}
{\answer{\valuedef{\omega}{0.98}{rad\,s\sup{-1}}}
The key point is to notice that angular momentum, $\vtr{J}$, must be conserved. Initially
\begin{equation*}
J=I_0\omega_0
\end{equation*}
where $I_0$ is the moment of inertia of the disc, $0.1\textrm{ kg m}^2$, and $\omega_0$ is its initial angular speed, $1\textrm{ rad s}^{-1}$. After the addition of the pencil sharpener, the moment of inertia changes from $I_0$ to $I_0+mr^2$, where $m$ is the mass of the pencil sharpener, and $r$ is the distance from the centre of the disc. Therefore we can find the final angular speed $\omega_1$
\begin{align*}
I_0\omega_0&=\left(I_0+mr^2\right)\omega_1 \\
\Rightarrow \omega_1&=\frac{I_0\omega_0}{\left(I_0+mr^2\right)} \\
&=0.98\textrm{ rad s}^{-1}
\end{align*}
}
\end{problem} 