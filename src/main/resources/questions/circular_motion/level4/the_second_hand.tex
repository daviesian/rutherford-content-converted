%% ID: the_second_hand
%% TITLE: The Second Hand
%% TYPE: question
%% QUESTIONTYPE: scq
%% CONCEPTS: angular_circular
%% VIDEOS: A1987PIQ7l.mp4
%% LEVEL: 4
%% TOPIC: mechanics/circular
%% ORDER: 3

\begin{problem}[A1987PIQ7l] 
{The second hand of a large clock is \quantity{3.0}{m} long. What is its mean angular speed?
\begin{enumerate}
	\item \quantity{1.4 \times 10^{-4}}{rad\,s\sup{-1}}
	\item \quantity{1.7 \times 10^{-3}}{rad\,s\sup{-1}}
	\item \quantity{5.2 \times 10^{-3}}{rad\,s\sup{-1}}
	\item \quantity{1.0 \times 10^{-1}}{rad\,s\sup{-1}} \answer
	\item \quantity{3.0 \times 10^{-1}}{rad\,s\sup{-1}}
\end{enumerate}}
{\stress{Used with permission from UCLES, A Level Physics, June 1987, Paper 1, Question 7.}}
{The correct answer is (d). This follows from the fact that there are \vari{2\pi} radians in a full circle, and the hand goes round every \quantity{60}{s}. Hence the mean angular speed is given by \valuedef{\omega}{\frac{2\pi}{60}}{} = \quantity{0.10472}{rad\,s\sup{-1}} $\approx$ \quantity{1.0 \times 10^{-1}}{rad\,s\sup{-1}}. We can also see some of the information is irrelevant as the radius of the hand has no bearing on its angular speed.}
\end{problem}