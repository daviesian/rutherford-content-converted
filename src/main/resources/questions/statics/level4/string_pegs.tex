%% ID: string_pegs
%% TITLE: Pegs and string
%% TYPE: question
%% QUESTIONTYPE:  scq
%% CONCEPTS: forces, newtoniii
%% VIDEOS: 
%% LEVEL: 4
%% TOPIC: mechanics/statics
%% ORDER: 2

\begin{problem}[A1985PIQ2]  %Diagram removed
{\question{A string is tied to two pegs P and R, with R above and to the right of P. A mass \vari{m} is attached to a point Q on the string such that the section PQ is horizontal and the section QR makes an angle of \quantity{30}{\sup{\circ}} to the horizontal. If the system is in equilibrium, what is the tension in the section of string PQ?
\begin{enumerate}
	\item \choice[a]{\vari{\frac{1}{4}mg}}
	\item \choice[b]{\vari{\frac{1}{2}mg}}
	\item \choice[c]{\vari{\frac{\sqrt{3}}{3}mg}}
	\item \choice[d]{\vari{\frac{\sqrt{3}}{2}mg}}
	\item \choice[e]{\vari{\sqrt{3}mg}}\correct
\end{enumerate}
}
}
{\textit{Used with permission from UCLES, A Level Physics, June 1985, Paper 1, Question 2.}}
{\answer{The correct answer is (e). Resolve forces on Q vertically to get \valuedef{mg}{T_{QR}\cos{60^{\circ}}}{}, then rearrange to get \valuedef{T_{QR}}{\frac{mg}{\cos{60^{\circ}}}}{}. Resolve horizontally to get \valuedef{T_{PQ}}{T_{QR}\sin{60^{\circ}}}{}. Substitute the expression for \vari{T_{QR}} to get \valuedef{T_{PQ}}{mg\tan{60^{\circ}}} = \vari{\sqrt{3}mg}.}
}
\end{problem}