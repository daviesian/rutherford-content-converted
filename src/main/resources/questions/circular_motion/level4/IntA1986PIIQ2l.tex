%% ID: record_player
%% TITLE: The Record Player
%% TYPE: question
%% QUESTIONTYPE: numeric
%% CONCEPTS: angular_circular
%% VIDEOS: IntA1986PIIQ2l.mov
%% LEVEL: 4
%% TOPIC: mechanics/circular
%% ORDER: 6

\begin{problem}[IntA1986PIIQ2l]
{A record is played at \quantity{45}{rpm}, and then slowed down to \quantity{33\frac{1}{3}}{rpm}. Find the ratio of the magnitude of the centripetal accelerations of a point on the rim of the record before and after the change. Write your answer to two decimal places.
} 
{\stress{Used with permission from UCLES, A Physics, November 1986, Paper 2, Question 2.}}
{Since the magnitude of the centripetal force required for circular motion is \vari{mR\omega^2} and \value{F}{ma}{}, the magnitude of the centripetal acceleration is \vari{R\omega^2}. So initially
\begin{equation*} a_0=R\omega_0^2 \end{equation*}
then 
\begin{equation*} a_1=R\omega_1^2 \end{equation*}
So the ratio of the first acceleration's magnitude \vari{a_0} to the second acceleration's magnitude \vari{a_1} is given by 
\begin{align*} \frac{a_0}{a_1}=\frac{\omega_0^2}{\omega_1^2} \end{align*}
Substituting in \nll
\value{\omega_0}{45}{rpm}= \quantity{45\cdot2\pi\cdot\frac{1}{60}}{rad\,s\sup{-1}} = \quantity{\frac{3}{2}\pi}{rad\,s\sup{-1}}\nll
and\nll
\value{\omega\s1}{33\frac{1}{3}}{rpm} = \quantity{33\frac{1}{3}\cdot2\pi\cdot\frac{1}{60}}{rad\,s\sup{-1}} = \quantity{\frac{10}{9}\pi}{rad\,s\sup{-1}}
\end{equation*}
finally
\begin{equation*} \frac{a_0}{a_1}=\frac{(3/2)^2}{(10/9)^2}=\frac{729}{400} \end{equation*}
}
\end{problem}