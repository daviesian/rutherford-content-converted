%% ID: skiers
%% TITLE: The Skiers
%% TYPE: question
%% QUESTIONTYPE: scq
%% CONCEPTS: energy
%% VIDEOS: 
%% LEVEL: 3
%% TOPIC: mechanics/dynamics
%% ORDER: 4

\begin{problem}[A1987PIQ5a] %Diagram removed
{Two skiers want to reach the top of an incline without pushing. The first skier, of mass $m$, reaches the start of the incline with a speed $v$. He just makes it to the top of the incline. The second skier, of mass $\frac{2}{3} m$, has a speed $\frac{2}{3}v$ at the bottom of the incline. Will she make it to the top without pushing? It can be assumed that frictional forces are negligible.
\begin{enumerate}
	\item No, she makes it to $\frac{8}{27}h$ 
	\item No, she makes it to $\frac{4}{9}h$ \answer
	\item No, she makes it to $\frac{8}{9}h$
	\item Yes, she just makes it to the top
	\item Yes, she makes it to the top with a non-zero velocity
\end{enumerate}
}
{\textit{Adapted with permission from UCLES, A Level Physics, June 1987, Paper 1, Question 5.}}
%{An object of mass $m$ passes a point X with a velocity $v$ and then slides up a frictionless incline to stop at a point Y which has a vertical height of $h$ above X. A second object of mass $\frac{1}{2}m$ passes X with a velocity of $\frac{1}{2}v$. To what height will it rise?
%\begin{enumerate}
%	\item $\frac{1}{8}h$ %    - A new answer since it has 1/8th the energy. May confuse some.
%	\item $\frac{1}{4}h$
%	\item $\frac{1}{2}h$
%	\item $\frac{1}{\sqrt{2}}h$
%	\item $h$
%%	\item $\sqrt{2} \; h$   - The original answer
%\end{enumerate}
%}
%{\textit{Used with permission from UCLES, A Level Physics, June 1987, Paper 1, Question 5.}}
{The correct answer is (b). This can be seen by conserving energy; the initial $h = \frac{v^{2}}{2g}$ and the second height would be %$h_{2} = \frac{v^{2}}{8g} = \frac{1}{4}h$.
$h_{2} = \frac{2v^{2}}{9} = \frac{4}{9}h$.
 Motion where gravitational potential is the only energy loss is independent of mass, since the $m$ cancels out in $\frac{1}{2}mv^{2} = mgh$.
% \\ \\ \\
%Initially $\frac{1}{2}mv^{2} = mgh$ or $h = \frac{v^{2}}{2g}$.\\ Using this, with $v_{0}$ and $h_{0}$ we obtain:\\ $h_{0} = \frac{v_{0}^{2}}{2g}$. \\ \\ When $v = \frac{2}{3}v_{0}$, we have $v^{2} = \frac{4}{9}v_{0}^{2}$. \\ \\Using the formula above, \begin{equation*}h = \frac{v^{2}}{2g} = \frac{v^{2}}{v_{0}^{2}}h_{0}\end{equation*} \\ From this it then follows: \begin{equation*} h = \frac{\left(\frac{4v_{0}^{2}}{9}\right)}{v_{0}^{2}} h_{0} = \frac{4}{9}h_{0} \end{equation*}
}
\end{problem}