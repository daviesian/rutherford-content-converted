%\input{../../../Templates/Problem_Template}
%\begin{document}
\begin{problem}[Space Justice] %A2-1
 {A police spaceship, of mass $m = 10,000$~kg, travelling at a speed $u = 2$~km~s$^{-1}$ needs to arrest another ship travelling ahead of it at 2.5~km~s$^{-1}$.  The police spaceship is capable of splitting itself into two equal parts and supplying them with kinetic energy from a single reserve of 2~GJ.  Is the reserve sufficient? \\
((If the reserve is sufficient; how long would it take for the police spaceship to catch the other craft, given that it was initially 3000 km behind?))
}
{\textit{Cambridge University Tripos 2011}}
{To solve this problem we must apply conservation of momentum and energy. Changing to the zero momentum frame we may say that the speed of each half of the spaceship is $v$ after the disintegration, conserving the overall momentum to be zero. Now we can calculate $v$ using the fact that there was 2~GJ of energy converted to the kinetic energy of the two parts. Since each part is travelling at $v$ in the ZMF, we may write:

\begin{equation*}2 \times \frac{1}{2}\frac{m}{2}v^2=2~GJ\end{equation*}

This yields $v=632ms^{-1}$. Converting back to the original frame we find that the velocity of the faster part of the spaceship is $0.632kms^{-1}+2kms^{-1}=2.632kms^{-1}$ and since this is faster than the other ship, the police will eventually catch them. 

To determine the time it would take for the police spaceship to catch the other craft, we may consider the relative velocity: $(2.632-2.5)kms^{-1}=0.132kms^{-1}$. The time taken is then simply $t=\frac{3000}{0.132}=22600 s$ or about 6.3 hours. 


}
\end{problem}
%\end{document}