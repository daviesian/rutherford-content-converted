%% ID: airzooka
%% TITLE: airzooka
%% TYPE: question
%% QUESTIONTYPE: numeric
%% CONCEPTS: momentum, energy
%% VIDEOS: 
%% LEVEL: 1
%% TOPIC: mechanics/dynamics
%% ORDER: 9

\begin{problem}[Robin9]
{An airzooka can fire a pellet of air across a room by quickly firing a squirt of air out of an elasticated polythene bag. 
\begin{enumerate}
	\item If the volume of the air pellet is \quantity{2}{litres}, what is the mass if the density of air is \quantity{1.2}{kg\,m^{-3}}?
	\item What is its kinetic energy if it crosses a \quantity{6}{m} wide room in \quantity{0.4}{s}?
	\item What is the momentum of the air pellet?
	\item Before the air pellet is fired, it is stationary.  It consists of air molecules moving around randomly at about \quantity{500}{m\,s^{-1}} ( a little faster than the speed of sound).  If the average mass of one molecule is \quantity{5\times10^{-26}}{kg}, how many molecules are there in the \quantity{2}{litre} packet of air?
	\item What is the total kinetic energy of all of the molecules?
	\item What is the total momentum of all of the molecules?
	\item If the kinetic energy is calculated using the answer to part (a) for the mass of the pellet, and the \quantity{500}{m\,s^{-1}} speed of the molecules, do you obtain the same as the answer to part (e), the total kinetic energy of the molecules?
	\item Why can the answer to part (f), the total momentum, not be calculated using the mass of the pellet in part (a) and the \quantity{500}{m\,s^{-1}} average speed of the molecules?
\end{enumerate}
}
{\stress{Question by RWH}}
{}
\end{problem}