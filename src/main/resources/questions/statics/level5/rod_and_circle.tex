%% ID: rod_and_circle
%% TITLE: Rod and circle
%% TYPE: question
%% QUESTIONTYPE:  scq
%% CONCEPTS: forces, vectors1, newtoni, newtonii, newtoniii
%% VIDEOS: 
%% LEVEL: 5
%% TOPIC: mechanics/statics
%% ORDER: 4

\begin{problem}[A1989FMIIQ1a]%balancing forces
%diagram removed, then re added
{\exposition{A smooth uniform rod of mass \vari{m} and length \vari{2a} is freely hinged to the ground at the point A between two smooth parallel vertical walls. The rod's other end, B, rests against one of the walls and a uniform circular lamina of mass \vari{\frac{m}{2}} is in contact with the other wall at a point P and with the rod at a point Q, as shown in figure \ref{fig:Statics_rodandcircleq} below. The rod and the lamina rest in equilibrium in the same vertical plane which is perpendicular to both walls.
\begin{figure}[h]
\centering
\includegraphics[width=5cm]{../../../figures/Statics_rodandcircleq.eps}
\caption{}
\label{fig:Statics_rodandcircleq}
\end{figure}}
\question{Given that the rod is inclined at an angle \valuedef{\theta}{60^\circ}{} to the horizontal and that the distance \valuedef{AQ}{\frac{2}{3}a}{}, find the magnitude of the reaction force at B.}
\begin{enumerate}
\item \choice[a]{\vari{mg}}
\item \choice[b]{\vari{\frac{7\sqrt 3}{18}mg}}
\item \choice[c]{\vari{\frac{\sqrt 3}{2}mg}}
\item \choice[d]{\vari{\frac{\sqrt 3}{9}mg}}
\item \choice[e]{\vari{\frac{1}{2}mg}}
\end{enumerate}
}
{\textit{Used with permission from UCLES, A Level Further Maths, Syllabus C, June 1989, Paper II, Question 1}}
{\answer{The correct answer is (b).}

The forces acting on the system are shown in Figure \ref{fig:Statics_rodandcirclea}:
\begin{figure}[h]
\centering
\includegraphics[width=5cm]{Statics_rodandcirclea}
\caption{}
\label{fig:Statics_rodandcirclea}
\end{figure}
\\

It is important to remember that a fixed hinge can provide a force with components normal to and parallel with a surface. In comparison, a reaction force can only have one component, normal to the surface it rests upon.

Some combination of resolving forces and taking moments will, as usual, be necessary to solve this problem.

 First, resolve vertically on the whole system:
\begin{align*}
\frac{3}{2}mg=H\sin\varphi
\end{align*}
and then horizontally
\begin{align*}
N_P=N_B+H\cos\varphi
\end{align*}
We can also resolve forces on just the circle, since that too is in equilibrium.Horizontally we have
\begin{align*}
R\cos\theta&=\frac{1}{2}mg \\
\Rightarrow R&=mg
\end{align*}
using $\cos\theta=1/2$ since $\theta=60^\circ$. Vertically
\begin{align*}
R\sin\theta&=N_P \\
\Rightarrow N_P&=\frac{\sqrt 3}{2}mg
\end{align*}
Next take moments on the rod about the hinge:
\begin{align*}
(2a/3)R+(a)mg\cos\theta&=(2a)N_B\sin\theta \\
\Rightarrow \frac{2}{3}R+\frac{1}{2}mg&=\sqrt 3N_B \\
\Rightarrow 4mg+3mg&=6\sqrt 3N_B \\
\Rightarrow N_B&=\frac{7\sqrt 3}{18}mg
\end{align*}
}
\end{problem}