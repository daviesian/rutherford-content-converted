%% ID: newtonii
%% VIDEOS: NewtonII.mp4
%% QUESTIONS: a_toboggan, firing_the_rockets, starting_w_basics, the_lift, a_well, on_ice, coiled_rope, launching_a_rocket, hailstorm, javelin_thrower, powering_up, an_accident, falling_chain, lifting_rod, rapid_acceleration, spring_and_thread, the_hosepipe, three_particles, water_wheel, chain_through_tube, road_collision, 
%% CONCEPTS: newtoni, newtoniii, vectors, calculus
%% LEVEL: 2
%% TOPIC: mechanics/dynamics
%% TYPE: physics
%% TITLE: Newton's Second Law
%% ORDER: 20

% This is the template that sets out all of the Problems and produces the Exercise/Solution labels and numbering
% There are two classes of Exercise: "problem" which has a Question and Solution, and "hint" which has a Question, Hint and Solution

% This is the template that sets out all of the Problems and produces the Exercise/Solution labels and numbering
% There are two classes of Exercise: "problem" which has a Question and Solution, and "hint" which has a Question, Hint and Solution

% These are the packages to use in all documents, and the paper size to use:
\documentclass[a4paper,11pt]{article}
\usepackage[usenames,dvipsnames]{xcolor}
\usepackage[margin=1.5cm]{geometry}
\usepackage{amsmath}
\usepackage{amssymb}
\usepackage{color}
\usepackage{graphicx}
\usepackage{graphics}
\usepackage[margin=1.5cm]{geometry}
\usepackage{fancyhdr}
\usepackage{float}
\usepackage{lscape}
\usepackage[font={small},labelfont=bf]{caption}
\usepackage{ifthen}
\usepackage{enumitem}
\usepackage{subcaption}		%Allows grouped figures. The percentage sign after the first \end{subfigure} puts them side by side, omitting it puts one above the other.
\usepackage{graphicx,xcolor} 	%Allows the use of colour in the files
\usepackage{centernot} 		%Puts the / in a not equal to sign in the centre, use as \cnot{...}
\usepackage{comment} 		%Allows \begin{comment} .... \end{comment} to comment out bulk text.
\usepackage{etoolbox}		%Allows the boolean flags and the \toggletrue and \togglefalse commands
\usepackage{cancel}		%Allows the crossing out of terms in maths mode to show they cancel out
\usepackage{wrapfig}

%Packages for font choices
\usepackage{palatino}
\usepackage{mathpazo}


%Then where to find the graphics:
%WARNING -  relative to the TeX file being compiled - NOT this template!
		\graphicspath{{../Diagrams/}{Diagrams/}{./}} %This allows diagrams: {{As a sister folder to Latex}{A subdirectory of LaTeX}{Or just in LaTeX itself}}

% WARNING -  If you want the diagrams to be a sister folder to the LaTeX folder - pdflatex.exe sometimes needs an extra argument to cope with the "../" part; usually it can only cope with subdirectories as opposed to parent ones. If it refuses to compile and says it cannot find the diagrams, either add "--shell-escape" to the start of the arguments of pdflatex, OR move the diagrams to a subdirectory of the one containing the TeX files.
%In TeXworks, to add the extra argument, go to Edit -> Preferences -> Typesetting -> Processing Tools. Click on "pdfLaTeX" -> Edit -> "+" button, then type "--shell-escape" (without quotes) and press the up arrow twice so that it becomes top of the list.


%Then any custom commands written, along with shortcuts and variables:
% This document contains any custom commands, shortcuts and variables needed for the files to compile. It is called by "Problem_Template.tex" and so needs to be in the same directory.

%Defines vectors universally, for ease of editing and consistency.
\newcommand{\vtr}[1] {\mathit{\underline{\boldsymbol{#1}}}}

%Draws a big red box containing the text as in \ALERT{<TEXT HERE>}. For labelling draft copies with important notes.
\def\ALERT#1{\begin{center}\colorbox{red}{\hbox{\textcolor{black}{\textbf{#1}}}}\end{center}}

%Roman-style subscript; removes math-mode font.
\def\s#1{_\textrm{#1} }

%The operators in integrals and derivatives.
\def\d{\operatorname{d}\!}

%The Euler e should be in Roman font.
\def\e{\textrm{e}}

%The Rutherford title, to save typing and for consistency:
\def\Rutherford{Rutherford School Physics}
\def\Concepttitle#1{\noindent\textsc{\Rutherford\vspace{0.4cm}\\ \LARGE Physical Concept: \textbf{#1}}}
\def\Problemtitle#1{\noindent\textsc{\Rutherford\vspace{0.4cm}\\ \LARGE Website Problems: \textbf{#1}}}
\def\AddProblemtitle#1{\noindent\textsc{\Large \Rutherford ~ --- ~ Additional Problems\vspace{0.4cm}\\ \LARGE \textbf{#1}}}
%\def\AddProblemtitle#1{\noindent\textsc{\Rutherford\vspace{0.4cm}\\ \LARGE Additional Problems: \textbf{#1}}}

%define quick question to be used in eg concept sheet.
%\def\qq#2{#1}{\color{red}[#2]\color{black}}
\newcommand{\qq}[2]{\nl Quick Question:\hspace{1 mm} #1\color{red}\hspace{2 mm} Answer:\color{black}\hspace{1 mm}  #2}
\newcommand{\stress}[1]{\emph{#1}}

%%%%%%%%%%%   some definitions used in latexing the CQMP:
% fractions that are of right size in set equations
\def\half{{\textstyle \frac{1}{2}}}
\def\quarter{{\textstyle \frac{1}{4}}}
\def\third{{\textstyle \frac{1}{3}}}
\def\eighth{{\textstyle \frac{1}{8}}}

% obtain a new line
\def\nl{\hfil\break}
\def\nll{\\ \\ \noindent}



%creates numbered lists with a), then i.
\renewcommand{\theenumi}{\alph{enumi}}% first level are latin characters
\renewcommand{\labelenumi}{\theenumi)} %tells it to put a bracket after the character.
\renewcommand{\theenumii}{\roman{enumii}}%second level are little roman characters
\renewcommand{\labelenumii}{\theenumii.} %tells it to put a dot after the character
 % In a file called "Definitions.tex" in the same directory as this file.

%Define some boolean switches:
\newtoggle{solutions_only}	%Print only the solutions
\newtoggle{no_solutions}		%Don't print any solutions  (overridden by solutions_only)
\newtoggle{solutions_at_end}	%Print the solutions at end (overridden by solutions_only and no_solutions)
\newtoggle{no_credits}		%Don't print the credit arguments

%Use this to write a list of things needed to know for a section. It automatically won't print when "solutions_only" is on.
%Its only argument should be a list of things needed to know in "\item [....]" form
\newenvironment{knowledge}[1]{
\iftoggle{solutions_only}{}{It is assumed that students will be familiar with the following concepts:
\begin{itemize} #1 \end{itemize}
\vspace{0.5cm}}
}

%Allows the headings to be managed when not printing problems ect.
\newenvironment{Qsection}[1]{
%\iftoggle{solutions_only}{}{\section{#1}} %Don't output headings in the solutions(?)
\iftoggle{solutions_at_end}{\AtEndDocument{\section{#1}}}{}
\section{#1}
}

\newenvironment{Qsubsection}[1]{
%\iftoggle{solutions_only}{}{\subsection{#1}} %Don't output headings in the solutions(?)
\iftoggle{solutions_at_end}{\AtEndDocument{\subsection{#1}}}{}
\subsection{#1}
}

%Set the values of the boolean switches: Yes - "toggletrue", No - "togglefalse".
\togglefalse{solutions_only}	%	ONLY		Output only solutions? 
\togglefalse{no_solutions}		%	NONE		Don't output solutions at all? 
\togglefalse{solutions_at_end}	%	END		Output solutions at the end?
\togglefalse{no_credits}		%			Don't output the credit field
%All 8 cases have been tested; ONLY takes precedence, then NONE and finally END is lowest.


%##############################################################################################################
%											Then the bulk of the layout options:
%##############################################################################################################

\setlength{\topmargin}{-2cm}
%\setlength{\oddsidemargin}{0.5cm}
%\setlength{\evensidemargin}{0.5cm}


%##############################################################################################################


\newcounter{exercisenumber}%[chapter] %counter is set to zero when "chapter" appears
\def\theexercisenumber{\arabic{exercisenumber}}


\iftoggle{no_solutions}{}{ %Put a header at the end before the solutions, and reset the counter. Only if solutions are being printed AND at the end.
	\iftoggle{solutions_only}{}{
		\iftoggle{solutions_at_end}
			{\AtEndDocument{\newpage \part*{Solutions:} \setcounter{exercisenumber}{0} \setcounter{section}{0}}}{}
	}
}


%%%%%%%%%%%%%%%%%%%%%%%%%%%%%%%%%%%%%%%%%%%%%%%%%%%%%%%%%%%%%%%%%%%%%%%%%%%%%%%%%
%Creates \begin{problem}[label]{exercise_text}{source_text}{solution_text}\end{problem} command - the label argument is optional
%If put in, remember to put in [] brackets.  A label called label.ex will be generated.
\newenvironment{problem}[4][noref]{
 \refstepcounter{exercisenumber} %\refstepcounter allows you to reference to the exercise number
%
\iftoggle{solutions_only}{\hfil\break \textit{Solution}~\theexercisenumber:  #4}{ %If only solutions, just output solution.
	\noindent{\textit{Exercise}~\theexercisenumber:}
	\ifthenelse{\equal{#1}{noref}}{}{\label{#1.ex}} #2 %\vspace{0.3cm}
	\iftoggle{no_credits}{}{
			%\hfil\break {\small #3} \vspace{0.3cm} %This is the old line, replaced with the one below, without the ifthenelse statement; in case something goes wrong.
			\ifthenelse{\equal{#3}{}}{}{ {\tiny [#3]} \vspace{0.3cm}} %If the credit field is blank; don't bother printing it or the space for it.
	} %reference argument
%
	\iftoggle{no_solutions}{}{ %If the solutions aren't to be printed, do nothing.
		\iftoggle{solutions_at_end}
			{\AtEndDocument{\stepcounter{exercisenumber}\hfil\break \textit{Solution}~\theexercisenumber:  #4 \vspace{0.5cm}}} %If at the end: do this.
			{\hfil\break \textit{Solution}~\theexercisenumber:  #4} 	%Else leave in line as in TeX file.
	}
}

\vspace{0.2cm}}

%%%%%%%%%%%%%%%%%%%%%%%%%%%%%%%%%%%%%%%%%%%%%%%%%%%%%%%%%%%%%%%%%%%%%%%%%%%%%%%%%
%Creates \begin{hint}[label]{exercise_text}{hint_text}{source_text}{solution_text}\end{hint} command - the label argument is optional
%If put in, remember to put in [] brackets.  A label called label.ex will be generated.
\newenvironment{hint}[5][noref]{
 \refstepcounter{exercisenumber}
%
\iftoggle{solutions_only}{\hfil\break \textit{Solution}~\theexercisenumber:  #5}{  %If only solutions, just output solution.
	\noindent{\textit{Exercise}~\theexercisenumber:}
	\ifthenelse{\equal{#1}{noref}}{}{\label{#1.ex}} #2 \vspace{0.1cm}
	 \hfil\break  \textit{Hint:}  #3{} %\vspace{0.3cm}
	\iftoggle{no_credits}{}{
			\ifthenelse{\equal{#4}{}}{}{\\ \hfil {\tiny #4} \vspace{0.3cm}} %This is the old line, replaced with the one below, without the ifthenelse statement; in case something goes wrong.
			%\ifthenelse{\equal{#4}{}}{}{{\tiny [#4]} \vspace{0.3cm}} %If the credit field is blank; don't bother printing it or the space for it.
	} %reference argument
%
	\iftoggle{no_solutions}{}{%If the solutions aren't to be printed, do nothing.
		\iftoggle{solutions_at_end}
			{\AtEndDocument{\stepcounter{exercisenumber}\hfil\break \textit{Solution}~\theexercisenumber:  #5 \vspace{0.5cm}}} %If at the end: do this.
			{\hfil\break \textit{Solution}~\theexercisenumber:  #5}%Else leave in line as in teX file.
	}
}
\vspace{0.2cm}}

%%%%%%%%%%%%%%%%%%%%%%%%%%%%%%%%%%%%%%%%%%%%%%%%%%%%%%%%%%%%%%%%%%%%%%%%%%%%%%%%%


 %%%%%%%
\newenvironment{additional}[2][noref]{
 \refstepcounter{exercisenumber}
% \vspace{.2cm}
\nl
\noindent{\textit{Exercise}~\theexercisenumber:}
\ifthenelse{\equal{#1}{noref}}{}{\label{#1.ex}} #2 }{%\vspace{5.1cm}
 }
%











% to mark as a draft.  Comment both these lines out when complete.  The page number will return to the footer.
%\pagestyle{myheadings}
%\markright{\textcolor{red}{\textbf{DRAFT: \today}}}


\begin{document}
\addtolength{\topmargin}{-0.7 cm}
\setlength{\columnsep}{22pt}
\Concepttitle{Newton's Second Law of Motion}

\section{As Newton stated it...}
The change of momentum of a body is proportional to the impulse applied to that body, and happens along the straight line on which the impulse is applied.
\begin{equation}
Impulse=\vtr{I} = \Delta{\vtr p} = \Delta(m \vtr{v}) \label{eqn:NII}
\end{equation}

\section{At school we commonly use this as...}
\begin{equation}
{\vtr F}=m{\vtr a}
\end{equation}
but we should be careful as this assumes that the mass of the object remains constant.  How do we get to this statement from Newton's original description (expressed in equation \ref{eqn:NII})?  What does this law tell us if we do not make any assumptions about the mass?
\subsection*{Levels 1-4}
The change in momentum depends on both the mass of the object and the velocity of the object and therefore it is possible in the real world that both of these could change due to the application of a resultant force. In the simplest cases however, we assume that the mass remains the same (does not change with time) and that it is only the velocity that is changing.
\\
\\
Let us begin with Newton's Second Law as he expressed it but convert it instead to a more familiar form, in terms of an applied force rather than an applied impulse, and \stress{assuming that the mass of our object is constant}.
\begin{eqnarray}
{\vtr I}&=&\Delta({m {\vtr v}}) =m(\Delta{\vtr v})\ \ \ (\mbox{assuming mass is constant})\nonumber\\
{\vtr F}\Delta t&=&m(\Delta{\vtr v}) \nonumber \\
{\vtr F}&=&\frac{m\Delta({\vtr v})}{\Delta t} \label{eqn:NIIF}
\end{eqnarray}
And with one further realisation that $\frac{\Delta {\vtr v}}{\Delta t}$  is an acceleration
\begin{equation}
{\vtr F}=m {\vtr a}
\end{equation}

\subsection*{Quick Questions}
\qq{If a resultant force of \quantity{20}{N} acts horizontally on a mass of \quantity{10}{kg}, what will be the resulting acceleration, \vari{a}?}{${\vtr a}={2}$ {m\,s\sup{-2}}}\nl
\qq{A block of mass \vari{M} is being pulled along a frictional surface by a rope with tension, \vari{\vtr{T}}.  The opposing frictional force due to the surface is \vari{\vtr F_{r}}.  What is the acceleration, \vari{\vtr a}, of the block?}  {
\begin{equation}
{\vtr a}=\frac{{\vtr T}-{\vtr F_r}}{M}\nonumber
\end{equation}}
\qq{As a train enters a station the buffers bring the train gently to rest from a speed of \quantity{2}{m\,s\sup{-1}} over \quantity{5}{s}.  What force, \vari{\vtr F}, must these buffers apply to a train of mass \quantity{5000}{kg}, and how does the direction of the force compare with the direction of the initial speed of the train?} {${\vtr F}={2000}$ {N}, in the opposite direction to the initial velocity of the train.}


\subsection{Level 5+}
When considering Newton's Second Law for a system in which a force is being applied to an object of changing mass, we need to consider not only the change in velocity over time but also the change in mass with time.  Let us consider, for example, a rocket in outer space that is burning fuel at a constant rate to produce a constant force of propulsion.  Here, mass is being continually lost and the velocity of the rocket is continuously changing due to the resulting constant force - so we have both changing mass and changing velocity.\\
\\
These changes are \stress{continuous} and therefore we can no longer represent Newton's Second Law in terms of large changes in time, \vari{\Delta t}, but instead have to consider how the mass and velocity change in infinitesimal amounts of time.  So
\begin{equation}
{\vtr F}= \frac{\delta{\vtr p}}{\delta t}= \frac{\delta(m {\vtr v})}{\delta t}.
\end{equation}
Now taking the limit as \vari{\delta t \rightarrow 0}, Newton's Second Law can be expressed in its derivative form
\begin{equation}
{\vtr F}=\frac{\d{\vtr p}}{dt}=\frac{\d(m{\vtr v})}{\d t} \label{NII2}
\end{equation}
Equation \ref{NII2} states that the resultant applied force causes an instantaneous rate of change of momentum, given by the derivative of the momentum with respect to time.  Momentum is the product of mass and velocity and therefore we must differentiate each of the terms, with respect to time, using the product rule.
 \begin{equation}
{\vtr F}=\frac{\d(m{\vtr v})}{\d t}={\vtr v}\frac{\d m}{\d t}+m\frac{\d{\vtr v}}{\d t}
\end{equation}
%\begin{wrapfigure}{r}{7cm}
%\center
%\includegraphics[width=0.35\textwidth]{NII-A.eps}
%\caption{The free-body force diagram for a chain falling under gravity on to the top of a table. ${\vtr W_{hang}}$ is the weight of the hanging chain, ${\vtr W}$ is the weight of the chain that rests on the table, ${\vtr N}$ is the normal reaction force and ${\vtr F_{dm}}$ is the force exerted by the table to bring the small, falling element of chain,$dm$, to rest. }\label{fig:chain} \vspace{2cm}
%\end{wrapfigure}
%{\vtr For example:} Imagine a chain falling onto the surface of a table at some time, $t$, such that some chain already lies on the table.\\
%\\
%In figure \ref{fig:chain} we show the free-body force diagram for the chain and those forces on the table due to both the weight of the chain and the force due to a falling element of the chain of mass, $dm$.  The chain already on the table is at rest and therefore we know from Newton's First Law that,
%\begin{equation}
%{\vtr  N}=W=m\frac{\d {\vtr v}}{\d t}=m{\vtr a}=m{\vtr g}
%\end{equation}
%In addition to this weight, the table must continually bring each small element of the chain to rest from a velocity $v$, immediately before it hits the table.  Thus the force on the table due to the rate of change of momentum of each small mass element is,
%\begin{equation}
%{\vtr F_{dm}}={\vtr v}\frac{\d m}{\d t}
%\end{equation}
%Here it is the mass that is changing with time as the velocity of each element is always, ${\vtr v}$, immediately before it hits the table.\\
%\\
%So, the total force on the table due to the chain is
%\begin{equation}
%{\vtr F_{total}}={\vtr F_w}+{\vtr F_{dm}}={\vtr v}\frac{\d m}{\d t}+m{\vtr g}
%\end{equation}
\end{document}
