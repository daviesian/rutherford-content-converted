\begin{problem}[Tripos_Ruler_Length]
{A an astronaut in a shuttle orbiting Earth notices a ruler floating in the capsule. She gives one end of it a sharp tap in a direction perpendicular to its length. Consequently its centre of mass moves away at a speed of $0.4\textrm{ ms}^{-1}$ and it performs a full rotation every $0.8\textrm{ s}$. What is the length of the ruler? }
{%\textit{Used/Adapted with permission from UCLES, A/O Level Physics/Maths/Physical Science, June/November 198, Paper, Question.}
}
{
Figure \ref{fig:AngularMotion_astro} shows the impulse $\Delta p$ being applied at one end of the rod:
\begin{figure}[h]
\centering
\includegraphics[width=7cm]{AngularMotion_astro}
\caption{}
\label{fig:AngularMotion_astro}
\end{figure}
\\
Since the rod is stationary before the tap, all of its linear and angular momentum aferwards will be due to the impulse of magnitude $\Delta p$. So 
\begin{equation*}
\Delta p=mv
\end{equation*}
and
\begin{equation*}
\frac{l}{2}\Delta p=I\omega=\frac{1}{12}ml^2\omega
\end{equation*}
for centre of mass velocity $v$, angular speed $\omega$. Substituting for $\Delta p$ in the second equation we find
\begin{align*}
\frac{l}{2}mv=\frac{1}{12}ml^2\omega \\
\Rightarrow l=\frac{6v}{\omega}
\end{align*}
Using $\omega=2\pi/T$
\begin{align*}
l=\frac{3vT}{\pi}=0.3\textrm{ m}
\end{align*}
}
\end{problem}