%% ID: acceleration_shm
%% TITLE: Acceleration during SHM
%% TYPE: question
%% QUESTIONTYPE: numerical
%% CONCEPTS: shm
%% VIDEOS: 
%% LEVEL: 4
%% TOPIC: mechanics/shm
%% ORDER: 5

%Physics: SHM, omega, period
%Maths: 
\begin{problem}[IntA1985PIIIQ1l]
{\exposition{A mass suspended from one end of a light spring undergoes vertical simple harmonic motion with an amplitude of \quantity{2.0}{cm}. Three complete oscillations are made in \quantity{4.0}{s}. 
\begin{enumerate}
\item \question[a]{Find the acceleration of the mass at the equilibrium position}
\item \question[b]{Find the acceleration of the massat the position of maximum displacement.}
\end{enumerate}
}
{\textit{Used with permission from UCLES, A Level Physics, November 1985, Paper 3, Question 1.}}
{
\begin{enumerate}
\item If the mass makes three complete oscillations in \quantity{4}{s}, then the period \vari{T}, the time taken for one complete oscillation, is \quantity{\frac{4}{3}}{s}. To find the angular frequency \vari{\omega} we can use the equation $\omega = \frac{2\pi}{T}$, so \valuedef{\omega}{\frac{3\pi}{2}}{s\sup{-1}}. We can then relate the acceleration to the displacement using $a = -\omega^2x$. \answer[a]{At the equilibrium position $x = 0$, so $a = 0$.}
\item At the position of maximum displacement \vari{x} is equal to the amplitude \vari{A}, so \valuedef{x}{0.02}{m}. Therefore $a = -\frac{9\pi^2}{4}\times0.02 = -0.444$, so \answer[b]{\valuedef{a}{0.444}{m\,s\sup{-2}} in the opposite direction to the displacement.}
\end{enumerate}
}
\end{problem}