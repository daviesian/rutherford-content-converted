%% ID: sprinter_100m
%% TITLE: Sprinter running 100m
%% TYPE: question
%% QUESTIONTYPE: numeric
%% CONCEPTS: newtonii, energy
%% VIDEOS: 
%% LEVEL: 1
%% TOPIC: mechanics/dynamics
%% ORDER: 5

\begin{problem}[Robin5]
{An \quantity{80}{kg} sprinter runs a \quantity{100}{m} race in \quantity{10.4}{s}. He accelerates to \quantity{9}{m\,s^{-1}} in the first \quantity{1.2}{s} with constant acceleration (this is a simplified model).
\begin{enumerate}
	\item What average force does he push with to get up to speed?
	\item What is the ratio of his accelerating force to his weight, which is supported by his legs?
	\item What distance does he travel in the first \quantity{1.2}{s}?
	\item What energy does he convert in \quantity{1.2}{s}?
	\item What average power do his legs produce to accelerate him to his racing speed?
	\item Since his legs can easily hold up more than twice his weight (he could easily lift a fellow athlete on to his shoulders) why is it that he cannot accelerate much faster at the start of the race?
\end{enumerate}
}
{\stress{Question by RWH}}
{}
\end{problem}