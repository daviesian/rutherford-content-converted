% This document contains any custom commands, shortcuts and variables needed for the files to compile. It is called by "Problem_Template.tex" and so needs to be in the same directory.

%Defines vectors universally, for ease of editing and consistency.
\newcommand{\vtr}[1] {\mathit{\underline{\boldsymbol{#1}}}}

%Draws a big red box containing the text as in \ALERT{<TEXT HERE>}. For labelling draft copies with important notes. This will be removed if included in a document for the website.
\def\ALERT#1{\begin{center}\colorbox{red}{\hbox{\textcolor{black}{\textbf{#1}}}}\end{center}}

%text-style subscript; removes math-mode font.
%\def\s#1{$_\text{#1}$ }

%The operators in integrals and derivatives.
\def\d{\operatorname{d}\!}

%The Euler e should be in the text font.
\def\e{\text{e}}

%Writes the superscript in the text font
\def\sup#1{$^\text{#1}$}
 
%Quantity is used for numbers with units: 1 kg, 1 m/s
\def\quantity#1#2{{${#1}$}{$\text{\,#2}$}}

%Vari puts a defined variable in maths font
\def\vari#1{${#1}$}

%Value is to be used when defining a variable with a quantity. If write \value{m}{1}{kg}, it will display m = 1 kg.
\def\value#1#2#3{\vari{#1} \quantity{=#2}{$\text{\,#3}$}}

% obtain a new line
\def\nl{\hfil\break}
\def\nll{\\ \\ \noindent}

%The Rutherford title, to save typing and for consistency: note: The text in these titles will be ignored for the website.
\def\Rutherford{Rutherford School Physics}
\def\Concepttitle#1{\noindent\textsc{\Rutherford\vspace{0.4cm}\\ \LARGE Physical Concept: \textbf{#1}}}
\def\ConceptMath#1{\noindent\textsc{\Rutherford\vspace{0.4cm}\\ \LARGE Mathematical Concept: \textbf{#1}}}
\def\Problemtitle#1{\noindent\textsc{\Rutherford\vspace{0.4cm}\\ \LARGE Website Problems: \textbf{#1}}}
\def\AddProblemtitle#1{\noindent\textsc{\Large \Rutherford ~ --- ~ Additional Problems\vspace{0.4cm}\\ \LARGE \textbf{#1}}}
%\def\AddProblemtitle#1{\noindent\textsc{\Rutherford\vspace{0.4cm}\\ \LARGE Additional Problems: \textbf{#1}}}

%define quick question to be used in eg concept sheet.
%\def\qq#2{#1}{\color{red}[#2]\color{black}}
\newcounter{qqnum}
\def\qqnumber{\arabic{qqnum}}
\newcommand{\qq}[2]{\noindent \stepcounter{qqnum}\qqnumber.~#1 \nll \stress{Answer:}\hspace{1 mm}  #2\nl}
\newcommand{\stress}[1]{\emph{#1}}

%define example environment to be used in eg concept sheet where examples are given rather than quick questions for the student to answer
%\def\examp#2{#1}{\color{red}[#2]\color{black}}
\newcommand{\examp}[2]{\nl {\bf For example:}\hspace{1 mm} #1\nll \stress{Solution:}
#2\nl}

%%%%%%%%%%%   some definitions used in latexing the CQMP:
% fractions that are of right size in set equations
\def\half{{\textstyle \frac{1}{2}}}
\def\quarter{{\textstyle \frac{1}{4}}}
\def\third{{\textstyle \frac{1}{3}}}
\def\eighth{{\textstyle \frac{1}{8}}}


%creates numbered lists with a), then i.
\renewcommand{\theenumi}{\alph{enumi}}% first level are latin characters
\renewcommand{\labelenumi}{\theenumi)} %tells it to put a bracket after the character.
\renewcommand{\theenumii}{\roman{enumii}}%second level are little roman characters
\renewcommand{\labelenumii}{\theenumii.} %tells it to put a dot after the character
