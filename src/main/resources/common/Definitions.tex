% This document contains any custom commands, shortcuts and variables needed for the files to compile. It is called by "Problem_Template.tex" and so needs to be in the same directory.

%Defines vectors universally, for ease of editing and consistency.
\newcommand{\vtr}[1] {\mathit{\underline{\boldsymbol{#1}}}}

%Draws a big red box containing the text as in \ALERT{<TEXT HERE>}. For labelling draft copies with important notes.
\def\ALERT#1{\begin{center}\colorbox{red}{\hbox{\textcolor{black}{\textbf{#1}}}}\end{center}}

%Roman-style subscript; removes math-mode font.
\def\s#1{_\textrm{#1} }

%The operators in integrals and derivatives.
\def\d{\operatorname{d}\!}

%The Euler e should be in Roman font.
\def\e{\textrm{e}}

%The Rutherford title, to save typing and for consistency:
\def\Rutherford{Rutherford School Physics} 
\def\Concepttitle#1{\noindent\textsc{\Rutherford\vspace{0.4cm}\\ \LARGE Physical Concept: \textbf{#1}}}
\def\Problemtitle#1{\noindent\textsc{\Rutherford\vspace{0.4cm}\\ \LARGE Website Problems: \textbf{#1}}}
\def\AddProblemtitle#1{\noindent\textsc{\Rutherford\vspace{0.4cm}\\ \LARGE Additional Problems: \textbf{#1}}}

%define quick question to be used in eg concept sheet.
%\def\qq#2{#1}{\color{red}[#2]\color{black}}
\newcommand{\qq}[2]{#1\color{red} \hspace{2 mm} Answer:  #2\color{black}}

%%%%%%%%%%%   some definitions used in latexing the CQMP:
% fractions that are of right size in set equations
\def\half{{\textstyle \frac{1}{2}}}
\def\quarter{{\textstyle \frac{1}{4}}}
\def\third{{\textstyle \frac{1}{3}}}
\def\eighth{{\textstyle \frac{1}{8}}}

% obtain a new line
\def\nl{\hfil\break}
\def\nll{\\ \\ \noindent}



%creates numbered lists with a), then i.
\renewcommand{\theenumi}{\alph{enumi}}% first level are latin characters
\renewcommand{\labelenumi}{\theenumi)} %tells it to put a bracket after the character.
\renewcommand{\theenumii}{\roman{enumii}}%second level are little roman characters
\renewcommand{\labelenumii}{\theenumii.} %tells it to put a dot after the character 