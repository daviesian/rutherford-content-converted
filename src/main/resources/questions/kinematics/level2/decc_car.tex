%% ID: decc_car
%% TITLE: A Uniformly Decelerating Car
%% TYPE: question
%% QUESTIONTYPE: scq
%% CONCEPTS: eq_of_motions_diff
%% VIDEOS: 
%% LEVEL: 2
%% TOPIC: mechanics/kinematics
%% ORDER: 2

\begin{problem}[A1985PIQ3l] 
{\exposition{The velocity of a car which is decelerating uniformly changes from \quantity{30}{m\,s\sup{-1}} to \quantity{15}{m\,s\sup{-1}} in \quntity{75}{m}}.\question{After what further distance will it come to rest?}
\begin{enumerate}
	\item \choice[a]{\quantity{25}{m}}\correct
	\item \choice[b]{\quantity{25}{37.5 m}}
	\item \choice[c]{\quantity{25}{50 m}}
	\item \choice[d]{\quantity{25}{75 m}}
	\item \choice[e]{\quantity{25}{100 m}}
\end{enumerate}
}
{\textit{Used with permission from UCLES, A Level Physics, June 1985, Paper 1, Question 3.}}
{\answer{The answer is (a). The equation $v^{2} = u^{2} + 2as$ can be used to find the value of the car's deceleration. Rearranging gives $a = \frac{v^{2} - u^{2}}{2s} = \frac{225 - 900}{150} =$ \quantity{-4.5}{m\,s\sup{-2}}. We can then use the equation again to find the distance travelled until the car comes to rest, where \valuedef{u}{15}{m\,s\sup{-1}} and \valuedef{v}{0}{m\,s\sup{-1}}. Rearrange the equation again to give $s = \frac{v^{2} - u^{2}}{2a} = \frac{-225}{-9} =$ \quantity{25}{m}.}
}
\end{problem}