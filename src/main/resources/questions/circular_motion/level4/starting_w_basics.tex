%% ID: starting_w_basics
%% TITLE: Starting with the Basics
%% TYPE: question
%% QUESTIONTYPE: scq
%% CONCEPTS: angular_circular, newtonii, vectors, resolving_vectors
%% VIDEOS: AO1989PIQ10l.mov
%% LEVEL: 4
%% TOPIC: mechanics/circular
%% ORDER: 1
\begin{problem}[AO1989PIQ10l] 
{\question{When a particle travels in a circle of radius \vari{r} with constant speed \vari{v}, what is its acceleration?}
\begin{enumerate}
	\item \choice[a]{\vari{\frac{v^{2}}{r}} towards the centre.}\correct
	\item \choice[b]{\vari{\frac{v^{2}}{r}} away from the centre.}
	\item \choice[c]{Zero}
	\item \choice[d]{\vari{rv^{2}} towards the centre.}
	\item \choice[e]{\vari{rv^{2}} away from the centre.}
\end{enumerate}
}
{\stress{Used with permission from UCLES, AO Level Physics, June 1989, Paper 1, Question 10.}}
{\answer{The correct answer is (a).} The centripetal force required to keep an object moving at a constant velocity in a circle is \vari{\frac{mv^{2}}{r}} towards the centre and so Newton's Second Law, \valuedef{\vtr{F}}{m\vtr{a}}{} says that the acceleration must therefore be \vari{\frac{v^{2}}{r}}, towards the centre also since \vari{\vtr{F}} and \vari{\vtr{a}} must lie in the same direction.}
\end{problem}