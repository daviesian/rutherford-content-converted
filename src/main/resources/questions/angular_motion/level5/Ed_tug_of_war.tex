\begin{problem}[Ed_tug_of_war]
{
\begin{figure}[h] 
\centering
\includegraphics[width=8cm]{AngularMotion_tug}
\caption{}
\label{fig:AngularMotion_tug}
\end{figure}
Two people are playing the game ``rotational tug of war". This involves two players at either end of a rod of mass $m=500\textrm{ kg}$ and length $l=5\textrm{ m}$ pivoted about its centre (which is fixed in the ground), both pulling as hard as they can, as shown in Figure \ref{fig:AngularMotion_tug} above. You win the game if you rotate the rod by an angle of $180^{\circ}$.
\begin{enumerate}
\item Player $1$ exerts a force of constant magnitude $F_1=300\textrm{ N}$, whilst player $2$ exerts a force of constant magnitude $F_2=350\textrm{ N}$. Find how long it takes player $2$ to win. 
\item For the next game player 3 joins in, and exerts a force of constant magnitude of $F_3= 150\textrm{ N}$ whilst standing halfway between the centre of the rod and player $1$. Find out whether players 1 and 3 win this time or player 2 wins again, and how long this game takes. 
\end{enumerate}
 }
{\textit{Written by Ed Scott and Zephyr Penoyre for the RSPP}}
{
\begin{enumerate}
\item The net moment on the rod, in the anti-clockwise direction, is given by 
\begin{equation*}
F_1l-F_2l=I\ddot\theta
\end{equation*}
The moment of inertia of a rod about its centre is $\frac{1}{12}ml^2$, as calculated earlier in the sheet. Therefore
\begin{align*}
F_1l-F_2l&=\frac{1}{12}ml^2\ddot\theta \\
\Rightarrow \ddot\theta&=\frac{12(F_2-F_1)}{ml}
\end{align*}
Now by analogy with the linear suvat equation $s=ut+\frac{1}{2}at^2$, we can form an equation to give the total angle rotated through:
\begin{equation*}
\theta=\omega_0t+\frac{1}{2}\ddot\theta t^2
\end{equation*}
Since the initial angular speed $\omega_0$ is zero, we can calculate $t$ directly:
\begin{align*}
\theta&=\frac{1}{2}\cdot\frac{12(F_1-F_2)}{ml}t^2 \\
\Rightarrow t^2&=\frac{ml\theta}{6(F_1-F_2)}
\end{align*}
Since $F_2>F_1$,  player 2 will win and we want the time it takes to reach $\theta=-\pi$ radians, so the final answer becomes
\begin{equation*}
t=\sqrt{\frac{-ml\pi}{6(F_1-F_2)}}=5.1\textrm{ s}
\end{equation*}
\item In this case the net torque on the rod in the clockwise direction is 
\begin{equation*}
F_1l+F_3l/2-F_2l=I\ddot\theta
\end{equation*}
Evaluating this gives a positive acceleration in the anti-clockwise direction, so this time players 1 and 3 will win. Using the exact same argument as in part a) but realising it will now reach an angle $+\pi$ radians, we find that the time taken is 
\begin{align*}
t=\sqrt{\frac{ml\pi}{6(F_1+F_3/2-F_2)}}=7.2\textrm{ s}
\end{align*}
\end{enumerate}
}
\end{problem}