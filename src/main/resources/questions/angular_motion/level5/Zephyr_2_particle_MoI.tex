\begin{problem}[Zephyr_2_particle_MoI] 
{Calculate the moments of inertia of two point masses with mass $m_1$ and $m_2$ separated by a distance $l$, rotating around their common centre of mass.
\begin{enumerate}
	\item $(m\s{1} + m\s{2})l^2$
	\item $\frac{m\s{1} + m\s{2}}{4} l^2$
	\item $\frac{m\s{1}~ m\s{2}}{(m\s{1} + m\s{2})} l^2$
	\item $\frac{m\s{1}^2~ m\s{2}^2}{(m\s{1} + m\s{2})^2} l^2$
	\item $(\frac{m\s{1}^2}{m\s{2}} + \frac{m\s{2}^2}{m\s{1}}) l^2$
\end{enumerate}}
{\textit{Written by Ed Scott and Zephyr Penoyre for the RSPP}}
{The correct answer is (c)
Use the formula that the centre of mass is given by 
\begin{equation*} \frac{\Sigma (\textrm{ mass}\times \textrm{distance})}{\Sigma \textrm{ mass}} \end{equation*}
to find that the distance of $m_1$ from the CoM is $\frac{m_2l}{m_1+m_2}$ and the mass $m_2$ is $\frac{m_1l}{m_1+m_2}$ from it. So using the formula $I=\Sigma mr^2$, we find 
\begin{equation*} I=m_1\left(\frac{m_2l}{m_1+m_2}\right)^2+m_2\left(\frac{m_1l}{m_1+m_2}\right)^2 \end{equation*}
which can be simplified to \begin{equation*}\frac{m\s{1}~ m\s{2}}{(m\s{1} + m\s{2})} l^2\end{equation*}
}
\end{problem}