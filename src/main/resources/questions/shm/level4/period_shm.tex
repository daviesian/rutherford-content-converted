%% ID: period_shm
%% TITLE: Period of SHM
%% TYPE: question
%% QUESTIONTYPE: scq
%% CONCEPTS: shm
%% VIDEOS: 
%% LEVEL: 4
%% TOPIC: mechanics/shm
%% ORDER: 2

\begin{problem}[A1986PIQ7l]
{\exposition{A particle moving with simple harmonic motion is measured to have an acceleration of \quantity{8}{mm\,s\sup{-1}} acting in one direction at an instant when its displacement is \quantity{2}{mm} in the opposite direction.} \question{What is the period of the motion?}
%Values of the acceleration $a$ of a particle moving in simple harmonic motion as a function of its displacement $x$ are given in the table below.
%\begin{table}[h]
%	\begin{center}
%	\begin{tabular}{lccccc}
%		$a$ / mm s$^{-2}$ \quad\quad\quad& 16 \quad\quad& 8 \quad\quad& 0 \quad\quad& -8 \quad\quad& -16\\
%		$x$ / mm \quad\quad\quad& -4 \quad\quad& -2 \quad\quad& 0 \quad\quad& 2 \quad\quad& 4\\
%	\end{tabular}
%	\end{center}
%\end{table}
%What is the period of the motion?
\begin{enumerate}
	\item \choice[a]{\quantity{1/\pi}{s}}
	\item \choice[b]{\quantity{2/\pi}{s}}
	\item \choice[c]{\quantity{\pi/2}{s}}
	\item \choice[d]{\quantity{2}{s}}
	\item \choice[e]{\quantity{\pi}{s}}
\end{enumerate}
}
{\textit{Used with permission from UCLES, A Level Physics, June 1986, Paper 1, Question 7.}}
{\answer{The correct answer is (e).} In simple harmonic motion, the acceleration of a particle is inversely proportional to its displacement from the origin, so $a = -\omega^{2}x$. By substituting the values for \vari{a} and \vari{x}, \vari{\omega} can be calculated; in this case, \valuedef{\omega}{2}{rad s\sup{-1}}$. The period of oscillations \valuedef{T}{\frac{2\pi}{\omega}}{}, so \valuedef{T}{\pi}{s}.
}
\end{problem}