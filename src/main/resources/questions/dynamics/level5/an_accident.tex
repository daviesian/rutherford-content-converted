%% ID: an_accident
%% TITLE: An Accident
%% TYPE: question
%% QUESTIONTYPE: scq
%% CONCEPTS: momentum, energy, newtoni, newtonii, impulse, resolving_forces
%% VIDEOS: 
%% LEVEL: 5
%% TOPIC: mechanics/dynamics
%% ORDER: 3

\begin{problem}[Phil_Car_Wall]
{\exposition{A car has a mass of \quantity{1000}{kg} and an engine capable of outputting a maximum power of \quantity{85}{kW}. When this car is travelling at a speed \vari{v} it experiences a resistive force equal to \quantity{1.6v^2}{N}.

The car is travelling at its maximum possible speed when it smashes into a brick wall, which brings it to an immediate halt.} \question{What impulse does the wall exert on the car?}
\begin{enumerate}
	\item \choice[a]{\quantity{25700}{kg\,m\,s\sup{-1}}}
	\item \choice[b]{\quantity{37600}{kg\,m\,s\sup{-1}}}\correct
	\item \choice[c]{\quantity{51400}{kg\,m\,s\sup{-1}}}
	\item \choice[d]{\quantity{75200}{kg\,m\,s\sup{-1}}}
	\item \choice[e]{\quantity{230000}{kg\,m\,s\sup{-1}}}
\end{enumerate}
}
{\stress{Created for the Rutherford School Physics Project by PS.}}
{\answer{The correct answer is (b).} At the car's maximum speed there is no acceleration, so the driving force is equal to the resistive force. \valuedef{P}{Fv}{}, so \valuedef{F}{\frac{P}{v}}{}; therefore \valuedef{\frac{85000}{v}}{1.6v^{2}}{}. From this you can work out \value{v}{37.6}{m\,s\sup{-1}}. The correct answer can be calculated using impulse \valuedef{I}{m\Delta v}{}.
}
\end{problem}

