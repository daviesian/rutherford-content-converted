%% ID: two_cubes
%% TITLE: Two Cubes
%% TYPE: question
%% QUESTIONTYPE: scq
%% CONCEPTS: newtonii
%% VIDEOS: 
%% LEVEL: 3
%% TOPIC: mechanics/dynamics
%% ORDER: 4

\begin{problem}[IntA1988PSIQ4a]  %Diagram removed.
\question{Two equal sized cubes, X and Y, of masses \vari{m} and \vari{2m} respectively, rest on a smooth horizontal plane with a face of one cube in contact with a face of the other. They are accelerated by a force \vari{F} applied to cube X, on the face opposite to that touching cube Y. What is the magnitude of the force exerted by block Y on block X during this acceleration?}
\begin{enumerate}
	\item \choice[a]{0}
	\item \choice[b]{\vari{\frac{1}{3}F}}
	\item \choice[c]{\vari{\frac{1}{2}F}}
	\item \choice[d]{\vari{\frac{2}{3}F}}\correct
	\item \choice[e]{\vari{F}}
\end{enumerate}
}
{\textit{Used with permission from UCLES, A Level Physical Science, November 1988, Paper 1, Question 4.}}
\answer{The correct answer is (d). This follows from Newton's Second Law, applied twice. Considering both cubes together, \valuedef{\vtr{F}}{m\vtr{a}}{} gives that \valuedef{F}{(m + 2m)a}{} so \valuedef{a}{\frac{F}{3m}}{} and then considering just cube X, and calling the force Y exerts on it \vari{R}, we have \valuedef{(F - R)}{ma}{} $=$ \vari{\frac{F}{3}} and so \valuedef{R}{\frac{2}{3}F}{}.}
\end{problem}