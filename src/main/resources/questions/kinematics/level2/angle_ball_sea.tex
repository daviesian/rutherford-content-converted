%% ID: angle_ball_sea
%% TITLE: A Ball Thrown into the Sea
%% TYPE: question
%% QUESTIONTYPE: scq
%% CONCEPTS: eq_of_motions_diff, forces, newtonii, trig, vectors1, vectors2
%% VIDEOS: 
%% LEVEL: 2
%% TOPIC: mechanics/kinematics
%% ORDER: 4

\begin{problem}[A1989PIQ3l] 
{\exposition{A projectile is fired horizontally, off a vertical cliff of height $h$, with horizontal velocity $v$.} \question{Which of the following values for $v$ and $h$ will give the steepest angle of impact when the projectile hits the sea at the base of the cliff?}\\

\quad\quad\quad $v$ / ms$^{-1}$ \quad\quad $h$ / m
\begin{enumerate}
	\item \choice[a]{\quad\quad\; 10 \quad\quad\quad\quad\quad 30}
	\item \choice[b]{\quad\quad\; 10 \quad\quad\quad\quad\quad 50}\correct
	\item \choice[c]{\quad\quad\; 30 \quad\quad\quad\quad\quad 30}
	\item \choice[d]{\quad\quad\; 30 \quad\quad\quad\quad\quad 50}
	\item \choice[e]{\quad\quad\; 50 \quad\quad\quad\quad\quad 10}
\end{enumerate}
}
{\textit{Used with permission from UCLES, A Level Physics, June 1989, Paper 1, Question 3.}}
{\answer{The correct answer is (b).} The angle of impact is the angle, $\theta$, the velocity vector makes with the ground at the instant the projectile hits. This means that $\tan(\theta) = \frac{v_{y}}{v_{x}}$, where $v_{x}$ is simply $v$ because there are no horizontal forces and $v_{y} = gt\s{impact}$. The time taken to hit the ground can be found simply using $h = \frac{1}{2}gt^{2}$ giving $t = \sqrt{\frac{2h}{g}}$, and so $v_{y} = \sqrt{2gh}$. This means that $\theta$ increases with $\tan(\theta) \propto \frac{\sqrt{h}}{v}$. The largest value of this comes from (b), and so it produces the largest angle of 72.3$^{\circ}$.
}
\end{problem}