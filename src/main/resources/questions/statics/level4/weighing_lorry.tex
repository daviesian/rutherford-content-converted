%% ID: weighing_lorry
%% TITLE: Weight of a lorry
%% TYPE: question
%% QUESTIONTYPE:  numerical
%% CONCEPTS: forces, moments, newtoni
%% VIDEOS: 
%% LEVEL: 4
%% TOPIC: mechanics/statics
%% ORDER: 5

\begin{problem}[HSC1930MIIIQ5a] % metric units added - numbers need to be made ``nice''; good problem as needs a few diagrams and a bit of thought - only two wheels on a platform makes it more interesting. 
{\exposition{A haulage company wants to determine the loaded weight of a lorry with four equal wheels, which is too large to stand with all four wheels on a single weighing platform.  
The lorry is placed so that the two front wheels are on one weighing platform and the two back wheels on another; the masses recorded are \quantity{1350}{kg} and \quantity{1450}{kg} respectively.  The axles are \quantity{3}{m} apart.  Find:}
\begin{enumerate}
 \item \question[a]{the mass of the loaded lorry.}
 \item \question[b]{the distance of the centre of gravity from the front axle.}
 \item \question[c]{what additional mass would have to be placed \quantity{50}{cm} in front of the front axle to make the weights borne by the axles equal.}
\end{enumerate}
}
{\textit{Adapted with permission from UCLES, Higher School Certificate Mathematics, June 1930, Paper 3, Question 5.}}
{\answer[a]{\valuedef{m}{2800}{kg}}
\answer[b]{\valuedef{d}{1.55}{m}}
\answer[c]{\valuedef{M}{75}{kg}}
Figure \ref{fig:Statics_lorry} shows the forces acting on the lorry whilst it is at rest on horizontal ground.
\begin{figure}[h]
\centering
\includegraphics[width=10cm]{Statics_lorry}
\caption{}
\label{fig:Statics_lorry}
\end{figure}
\\
\begin{enumerate}
\item The weighing platform tells us that $R_F=1350g$ and $R_B=1450g$ and we can resolve vertically to find
\begin{align*}
R_F+R_B&=mg \\
\Rightarrow1350g+1450g&=mg \\
\Rightarrow m&=2800\textrm{ kg}
\end{align*}
\item Taking moments about the rear wheel we have
\begin{align*}
(l)R_F&=(\alpha l)mg \\
\Rightarrow \alpha&=\frac{R_F}{mg}=\frac{27}{56}
\end{align*}
So the distance from the front axle is
\begin{align*}
(1-\alpha)l&=(1-\frac{27}{56})3=1.55\textrm{ m}
\end{align*}
\item There is now a new force and the forces on the wheels will have changed. This is shown in Figure \ref{fig:Statics_lorrytwo}:
\begin{figure}[h]
\centering
\includegraphics[width=10cm]{Statics_lorrytwo}
\caption{}
\label{fig:Statics_lorrytwo}
\end{figure}
\\
Once again take moments about the back wheel:
\begin{align*}
\left(\frac{7}{6}l\right)kmg+(\alpha l)mg&=(l)N \\
\Rightarrow \frac{7}{6}kmg+\alpha mg&=N
\end{align*}
and about the front wheel:
\begin{align*}
\left(\frac{1}{6}l\right)kmg+(l)N&=(1-\alpha)lmg \\
\Rightarrow \frac{1}{6}kmg+N&=(1-\alpha)mg
\end{align*}
Adding these two equations we find
\begin{align*}
\frac{7}{6}kmg+\alpha mg+\frac{1}{6}kmg+N&=N+(1-\alpha)mg \\
\Rightarrow \frac{4}{3}kmg&=(1-2\alpha)mg \\
\Rightarrow k&=\frac{3}{4}(1-2\alpha)=\frac{3}{112}
\end{align*}
So the mass required would be $3m/112=75$ kg.
\end{enumerate}
}
\end{problem}