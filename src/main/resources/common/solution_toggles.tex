%Define some boolean switches:
\newtoggle{solutions_only}	%Print only the solutions
\newtoggle{no_solutions}		%Don't print any solutions  (overridden by solutions_only)
\newtoggle{solutions_at_end}	%Print the solutions at end (overridden by solutions_only and no_solutions)
\newtoggle{no_credits}		%Don't print the credit arguments

%Use this to write a list of things needed to know for a section. It automatically won't print when "solutions_only" is on.
%Its only argument should be a list of things needed to know in "\item [....]" form
\newenvironment{knowledge}[1]{
\iftoggle{solutions_only}{}{It is assumed that students will be familiar with the following concepts:
\begin{itemize} #1 \end{itemize}
\vspace{0.5cm}}
}

%Allows the headings to be managed when not printing problems ect.
\newenvironment{Qsection}[1]{
%\iftoggle{solutions_only}{}{\section{#1}} %Don't output headings in the solutions(?)
\iftoggle{solutions_at_end}{\AtEndDocument{\section{#1}}}{}
\section{#1}
}

\newenvironment{Qsubsection}[1]{
%\iftoggle{solutions_only}{}{\subsection{#1}} %Don't output headings in the solutions(?)
\iftoggle{solutions_at_end}{\AtEndDocument{\subsection{#1}}}{}
\subsection{#1}
}

%Set the values of the boolean switches: Yes - "toggletrue", No - "togglefalse".
\togglefalse{solutions_only}	%	ONLY		Output only solutions? 
\togglefalse{no_solutions}		%	NONE		Don't output solutions at all? 
\togglefalse{solutions_at_end}	%	END		Output solutions at the end?
\togglefalse{no_credits}		%			Don't output the credit field
%All 8 cases have been tested; ONLY takes precedence, then NONE and finally END is lowest.


%##############################################################################################################
%											Then the bulk of the layout options:
%##############################################################################################################

\setlength{\topmargin}{-2.7cm}
%\setlength{\oddsidemargin}{0.5cm}
%\setlength{\evensidemargin}{0.5cm}


%##############################################################################################################


\newcounter{exercisenumber}%[chapter] %counter is set to zero when "chapter" appears
\def\theexercisenumber{\arabic{exercisenumber}}


\iftoggle{no_solutions}{}{ %Put a header at the end before the solutions, and reset the counter. Only if solutions are being printed AND at the end.
	\iftoggle{solutions_only}{}{
		\iftoggle{solutions_at_end}
			{\AtEndDocument{\newpage \part*{Solutions:} \setcounter{exercisenumber}{0} \setcounter{section}{0}}}{}
	}
}


%%%%%%%%%%%%%%%%%%%%%%%%%%%%%%%%%%%%%%%%%%%%%%%%%%%%%%%%%%%%%%%%%%%%%%%%%%%%%%%%%
%Creates \begin{problem}[label]{exercise_text}{source_text}{solution_text}\end{problem} command - the label argument is optional
%If put in, remember to put in [] brackets.  A label called label.ex will be generated.
\newenvironment{problem}[4][noref]{
 \refstepcounter{exercisenumber} %\refstepcounter allows you to reference to the exercise number
%
\iftoggle{solutions_only}{\hfil\break \textit{Solution}~\theexercisenumber:  #4}{ %If only solutions, just output solution.
	\noindent{\textit{Exercise}~\theexercisenumber:}
	\ifthenelse{\equal{#1}{noref}}{}{\label{#1.ex}} #2 %\vspace{0.3cm}
	\iftoggle{no_credits}{}{
			%\hfil\break {\small #3} \vspace{0.3cm} %This is the old line, replaced with the one below, without the ifthenelse statement; in case something goes wrong.
			\ifthenelse{\equal{#3}{}}{}{ {\tiny [#3]} \vspace{0.3cm}} %If the credit field is blank; don't bother printing it or the space for it.
	} %reference argument
%
	\iftoggle{no_solutions}{}{ %If the solutions aren't to be printed, do nothing.
		\iftoggle{solutions_at_end}
			{\AtEndDocument{\stepcounter{exercisenumber}\hfil\break \textit{Solution}~\theexercisenumber:  #4 \vspace{0.5cm}}} %If at the end: do this.
			{\hfil\break \textit{Solution}~\theexercisenumber:  #4} 	%Else leave in line as in TeX file.
	}
}

\vspace{0.2cm}}

%%%%%%%%%%%%%%%%%%%%%%%%%%%%%%%%%%%%%%%%%%%%%%%%%%%%%%%%%%%%%%%%%%%%%%%%%%%%%%%%%
%Creates \begin{hint}[label]{exercise_text}{hint_text}{source_text}{solution_text}\end{hint} command - the label argument is optional
%If put in, remember to put in [] brackets.  A label called label.ex will be generated.
\newenvironment{hint}[5][noref]{
 \refstepcounter{exercisenumber}
%
\iftoggle{solutions_only}{\hfil\break \textit{Solution}~\theexercisenumber:  #5}{  %If only solutions, just output solution.
	\noindent{\textit{Exercise}~\theexercisenumber:}
	\ifthenelse{\equal{#1}{noref}}{}{\label{#1.ex}} #2 \vspace{0.1cm}
	 \hfil\break  \textit{Hint:}  #3{} %\vspace{0.3cm}
	\iftoggle{no_credits}{}{
			\ifthenelse{\equal{#4}{}}{}{\\ \hfil {\tiny #4} \vspace{0.3cm}} %This is the old line, replaced with the one below, without the ifthenelse statement; in case something goes wrong.
			%\ifthenelse{\equal{#4}{}}{}{{\tiny [#4]} \vspace{0.3cm}} %If the credit field is blank; don't bother printing it or the space for it.
	} %reference argument
%
	\iftoggle{no_solutions}{}{%If the solutions aren't to be printed, do nothing.
		\iftoggle{solutions_at_end}
			{\AtEndDocument{\stepcounter{exercisenumber}\hfil\break \textit{Solution}~\theexercisenumber:  #5 \vspace{0.5cm}}} %If at the end: do this.
			{\hfil\break \textit{Solution}~\theexercisenumber:  #5}%Else leave in line as in teX file.
	}
}
\vspace{0.2cm}}

%%%%%%%%%%%%%%%%%%%%%%%%%%%%%%%%%%%%%%%%%%%%%%%%%%%%%%%%%%%%%%%%%%%%%%%%%%%%%%%%%


 %%%%%%%
\newenvironment{additional}[2][noref]{
 \refstepcounter{exercisenumber}
% \vspace{.2cm}
\nl
\noindent{\textit{Exercise}~\theexercisenumber:}
\ifthenelse{\equal{#1}{noref}}{}{\label{#1.ex}} #2 }{%\vspace{5.1cm}
 }
%