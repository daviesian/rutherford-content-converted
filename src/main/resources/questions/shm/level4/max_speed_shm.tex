%% ID: max_speed_shm
%% TITLE: Maximum speed in SHM
%% TYPE: question
%% QUESTIONTYPE: scq
%% CONCEPTS: shm
%% VIDEOS: 
%% LEVEL: 4
%% TOPIC: mechanics/shm
%% ORDER: 1

\begin{problem}[A1987PIQ10l]
{\exposition{A particle performs simple harmonic motion of amplitude \quantity{0.020}{m} and frequency\quantity{2.5}{Hz}.} \question{What is its maximum speed?}
\begin{enumerate}
	\item \choice[a]{\quantity{0.008}{ms\sup{-1}}}
	\item \choice[b]{\quantity{0.050}{ms\sup{-1}}}
	\item \choice[c]{\quantity{0.125}{ms\sup{-1}}}
	\item \choice[d]{\quantity{0.157}{ms\sup{-1}}}
	\item \choice[e]{\quantity{0.314}{ms\sup{-1}}}\correct
\end{enumerate}
}
{\textit{Used with permission from UCLES, A Level Physics, June 1987, Paper 1, Question 10.}}
{\answer{The correct answer is (e).} By differentiating the solution of the simple harmonic equation, $x = A\cos{\omega t}$, with respect to time, the expression $v = -A\omega\sin{\omega t}$ can be obtained. The maximum absolute value that the $\sin$ function can take is 1, so the maximum absolute value that \vari{v} can take is $A\omega$. $\omega = 2\pi f$, so $v\s{max} = 2\pi fA = 0.314$ ms$^{-1}$.
}
\end{problem}