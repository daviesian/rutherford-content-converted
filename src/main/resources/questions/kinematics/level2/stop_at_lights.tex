%% ID: stop_at_lights
%% TITLE: Stopping At The Lights
%% TYPE: question
%% QUESTIONTYPE: numeric
%% CONCEPTS: forces, newtonii, eq_of_motions_diff
%% VIDEOS: 
%% LEVEL: 2
%% TOPIC: mechanics/kinematics
%% ORDER: 7

\begin{problem} [Phil_car_stoplight]
{\exposition{A car is travelling along a road at constant speed $u = 7$ m s$^{-1}$. The driver sees a stoplight ahead change to amber and applies the brakes. Consequently, the car experiences a total resistive force equal to $\frac{1}{6}$ of its weight and stops just at the stoplight.} \question{Taking $g = 9.8$ m s$^{-2}$, how far does the car travel during the braking period?}
}
{\textit{Created for the Rutherford School Physics Project by PS.}}
{We are not told the mass of the car, so let's call it $m$. The resistive force experienced by the car $= \frac{1}{6}mg$. Using Newton's second law, the car's deceleration is therefore $\frac{1}{6}g$. We can now use the equation $v^2 = u^2 + 2as$, where $v = 0$ and $a = -\frac{1}{6}g$(as the car is decelerating).
\begin{equation*}	0 = u^2 - \frac{1}{6}gs	\end{equation*}
\begin{equation*}	u^2 = \frac{1}{6}gs	\end{equation*}
\begin{equation*}	\Rightarrow s = \frac{6u^2}{g}	\end{equation*}
Substituting numerical values:
\begin{equation*}	s = \frac{6\times 49}{9.8} = 30	\end{equation*}
\answer{The car travels 30 m while braking.}
}
\end{problem}