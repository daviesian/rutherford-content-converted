%% ID: stopping_distances
%% TITLE: Stopping Distances
%% TYPE: question
%% QUESTIONTYPE: numeric
%% CONCEPTS: eq_of_motions_diff, forces, newtonii, trig, vectors1
%% VIDEOS: 
%% LEVEL: 2
%% TOPIC: mechanics/kinematics
%% ORDER: 6


\begin{problem}[A1989PIIQ9l] %SUVAT
{\exposition{A table from a car driver's handbook reads as follows:
  
  \textit{On a dry road, a car in good condition driven by an alert driver will stop in the distances shown in the table.}
  \begin{table}[h]	 %[h] tells the table to be placed as close as possible to here.
	%\caption{Title}	 %This is the title
	\begin{center} 		%This centres the table.
	\begin{tabular}{|c|c|c|c|}	%This says there are 3 columns with a vertical line around all columns. 
%					%The text is centred in the columns. If it were to the right it would be | r |, or 
%					%left: | l |
	\hline
	\textit{Speed} /m s$^{-1}$ & \textit{Thinking* distance} /m & \textit{Braking distance$^{\dagger}$} /m & \textit{Overall distance} /m \\
	\hline
	5.0 & 3.0 & 1.9 & 4.9\\
	10 & 6.0 & 7.5 & 13.5 \\
	15 & 9.0 & 17 & 26 \\
	20 & 12 & 30 & 42 \\
	25 & 15 & 47 & 62 \\
	30 & 18 & 68 & 86 \\
	35 & 21 & 92 & 113 \\
	\hline
	\end{tabular}
	\end{center}
	\label{table:sintering}
  \end{table}
  
  *The \textit{Thinking distance} is the distance travelled by the car during the driver's reaction time.
  
  $\dagger$The \textit{Braking distance} is the distance in which the car stops after the brakes have been applied.}
  
  \begin{enumerate}
  	\item \question[a]{Using equations, explain why thinking distance is directly proportional to speed whereas braking distance is not. (i.e. give the separate equations which govern thinking and braking distances)} %website input?
	\item \question[b]{What constant value of negative acceleration has the author of the table used in the calculating the braking distances?}
	\item \question[c]{Calculate the overall stopping distance for a car travelling at 50 m s$^{-1}$.}
	\item \question[d]{Will the thinking distance and the braking distance be effected when
	\begin{enumerate} %can have thinking: yes/no and braking: yes/no (multiple choice?) input for each question
		\item The road is wet?
		\item The driver is not fully alert?
	\end{enumerate}}
	
	\item \question[e]{Calculate the overall stopping distance for a car travelling at a speed of 35 m s$^{-1}$ down a hill at an angle of 10$^{\circ}$ to the horizontal.}
  \end{enumerate}
}
{\textit{Used with permission from UCLES A Level Physics, June 1989, Paper II, Question 9.}}
{\begin{enumerate}
\item \answer[a]{Thinking distance = speed $\times$ time \\
Breaking distance $= ut+\frac{1}{2}at^2$}

Explanation:
Thinking distance is just the distance travelled during the time it takes for the driver to realise he/she needs to brake and then apply the brakes. Therefore there is no acceleration and the thinking distance is equal to the speed of the car multiplied by the time taken for the driver to apply the brakes, i.e. the thinking distance is directly proportional to speed, whereas there is another term due to deceleration in the braking distance. (From $s=ut+at^2/2$). 
\item Use $v^2=u^2+2as$; $v=0$ since the car's final speed is zero. Using data from the first line, we find $a=-u^2/(2s)=6.58$ ms$^{-2}$. The 2nd, 3rd, 4th and 5th lines give values, in ms$^{-2}$, of 6.67, 6.62, 6.67 and 6.65. Using a value of 6.67 ms$^{-2}$ gives the right data to 2s.f. in each line, so we can assume that is the constant acceleration used.  \answer[b]{The negative acceleration is 6.67ms$^{-2}$.}
\item The thinking distance for a car travelling at 50 ms$^{-2}$ is 30 m, and the braking distance is given by $s=-u^2/(2a)=187.5$ m. \answer[c]{The overall stopping distance is 217.5 m.}
\item 
\begin{enumerate}
\item Thinking: no \\ Braking: yes \\ \\
Explanation: This will not affect the thinking distance (unless it's raining and the visibility is poor) but will increase the braking distance since the coefficient of friction between the tyres and the road will be lower, resulting in a slower deceleration once the brakes are applied. 
\item Thinking: yes \\ Braking: no \\ \\
This will increase the thinking distance since the driver will take longer to react but not affect the braking distance.
\answer[d]{The wet road will not affect the thinking distance (unless it's raining and the visibility is poor) but will increase the braking distance. The driver not being fully alert will increase the thinking distance since the driver will take longer to react but not affect the braking distance.}
\end{enumerate}
\item Thinking distance is 21 m, unaffected by the slope since we are assuming the car maintains a constant speed until the brakes are applied. The braking distance will increase since there is now a force due to the weight that acts down the slope, in the opposite direction to the force produced by the brakes. The component down the slope is $mg\sin(10)$ so on its own the force would produce an acceleration of $g\sin(10)=1.70$ ms$^{-2}$. Therefore the overall acceleration is now $6.67-1.70=4.97$ ms$^{-2}$. This means that the braking distance becomes $s=35^2/(2\times4.97)=123.3$ m. \answer[e]{The total stopping distance is 144.3 m.}
\end{enumerate}
}
\end{problem}