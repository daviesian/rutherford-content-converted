%% ID: coiled_rope
%% TITLE: A Coiled Rope
%% TYPE: question
%% QUESTIONTYPE: numeric
%% CONCEPTS: energy, momentum, newtonii
%% VIDEOS: 
%% LEVEL: 4
%% TOPIC: mechanics/dynamics
%% ORDER: 7

\begin{problem}[HSC1938PIIX3a] 
{A coil of rope of length \value{l}{10}{m} and of mass {m}{1}{kg} is lying on a horizontal floor. A man takes hold of one end and walks off at a speed {v}{5}{m\,s\sup{-1}}, the rope uncoiling as he goes.
\begin{enumerate}
	\item Neglecting friction, calculate the force he exerts on the rope.
	\item When the whole rope has been started in motion, calculate:
	\begin{enumerate}
		\item the kinetic energy acquired,
		\item the work done by the man.
	\end{enumerate}
\end{enumerate}
Comment on your result.
}{
\stress{Adapted with permission from UCLES, Higher School Certificate Physics, June~1938, Paper~2.}
}{ %Question number missing - X
\begin{enumerate}
	\item If there is no friction, the only force is that required to start a new section of the rope moving. Using Newton's Second Law, \value{\vtr{F}}{\frac{\d p}{\d t}}{} which simplifies to \value{\vtr{F}}{\vtr{v}\frac{\d m}{\d t}}{} since \vari{v} is constant, we can find the force needed to start the rope moving. To do this we need \vari{\frac{\d m}{\d t}}, which can be found by considering the extra mass of rope added in a time \vari{\d t}:
\begin{equation*} 
\d m = (\text{Mass per unit length, }\lambda)\times(\text{Length added in } \d t) = \lambda v \d t 
\end{equation*}
and so it follows that: \value{\frac{\d m}{\d t}}{\lambda v}{}.
Hence:
\begin{eqnarray*} 
F &= v\frac{\d m}{\d t} \\ 
&= \lambda v^{2} \\ 
&= \frac{mv^{2}}{l} \\ 
&= \frac{(1)(5)^{2}}{(10)}\\ 
&= \mbox{\quantity{2.5}{N}}
\end{eqnarray*}
\item Once the whole rope is in motion, the rope is travelling at \vari{v} and the force is no longer needed:
	\begin{enumerate}
		\item The kinetic energy is simple:
			\begin{eqnarray*} 
\text{KE} &= \frac{1}{2}mv^{2} \\
&= \frac{1}{2}(1)(5)^{2} \\ 
&= \mbox{\quantity{12.5}{J}} 
            \end{eqnarray*}
		\item The work done for a constant force is \value{W}{\vtr{F}\cdot\vtr{x}}{} and so the work done by the man is:
				\begin{eqnarray*} 
			        W &= Fl \\ 
			        &= \left(\frac{mv^{2}}{l}\right)l \\ 
			        &= mv^{2} \\
			        &= (1)(5)^{2}\\ 
			        &= \mbox{\quantity{25}{J}} 
			    \end{eqnarray*}
	\end{enumerate}
\item The kinetic energy obtained is half the total work done by the man. The rest of the work done has been dissipated as heat.
\end{enumerate}
}
\end{problem}