%% ID: heavy_ball
%% TITLE: Falling heavy ball
%% TYPE: question
%% QUESTIONTYPE: slc
%% CONCEPTS: eq_of_motions, newtonii
%% VIDEOS: 
%% LEVEL: 1
%% TOPIC: mechanics/dynamics
%% ORDER: 4

\begin{problem}[Robin4]
{Using a value of \valuedef{g}{10}{m\,s^{-2}}, a heavy ball is dropped from rest. Ignoring air resistance, its speed after one, two and three seconds is
\begin{enumerate}
	\item 1 \quad 2 \quad 3 \quantity{}{m\,s^{-1}}
	\item 0 \quad 2 \quad 4 \quantity{}{m\,s^{-1}}
	\item 1 \quad 4 \quad 9 \quantity{}{m\,s^{-1}}
	\item 10 \quad 20 \quad 30 \quantity{}{m\,s^{-1}} \answer
	\item 1 \quad 4 \quad 8 \quantity{}{m\,s^{-1}}
\end{enumerate}
}
{\stress{Question by RWH}}
{The correct answer is (d).    g is the particular value of the acceleration due to the gravitational pull of the earth. If it is any old acceleration due to a force, then the letter a is used. Acceleration is the rate of change of velocity. So, starting from rest, after \quantity{1}{second} the velocity must be \quantity{10}{m\,s^{-1}} downwards, and after another second it will again have increased by another \quantity{10}{m\,s^{-1}}, making \quantity{20}{m\,s^{-1}}, and then to \quantity{30}{m\,s^{-1}} after the third second. }
\end{problem}