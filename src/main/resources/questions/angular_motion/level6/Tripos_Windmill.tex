\begin{hint}[Tripos_Windmill]
{
\begin{enumerate}
\item Define the impulse exerted when a force acts for a short time, and relate the impulse to the changes in linear and angular momenta during collisions. 
\end{enumerate}
A toy windmill consists of four thin uniform rods of mass $m$ and length $l$ arranged at right angles, in a vertical plane, around a horizontal axle about which they can rotate freely. 
\begin{enumerate}
\setcounter{enumi}{1}
\item Show that the moment of inertia of the windmill about the axle is $\frac{4}{3}ml^2$.
\end{enumerate}
Initially the windmill is stationary. A small ball of mass $m$ is dropped from a height $h$ above the axle, and makes an elastic collision with the end of a horizontal rod. 
\begin{enumerate}
\setcounter{enumi}{2}
\item Derive an expression for the angular speed of the windmill after the collision.
\item To what height does the ball rebound?
\end{enumerate}
}
{
\begin{enumerate}
\setcounter{enumi}{2}
\item Think about the quantities that will be conserved, given that the collision is elastic. 
\end{enumerate}
}
{%\textit{Used/Adapted with permission from UCLES, A/O Level Physics/Maths/Physical Science, June/November 198, Paper, Question.}
}
{
\begin{enumerate}
\item The impulse is, formally, the integral of the force over a short time interval. This means it is given by force multiplied by time if the force is constant. For an impulse of magnitude $\Delta P$ then the change in linear momentum is $\Delta P$ in the same direction as the impulse and the change in angular momentum has magnitude $r\Delta P$ and a direction perpendicular to the impulse and the direction of displacement from the axis of rotation. $r$ is the perpendicular distance from the point the impulse is applied to the axis of rotation. 
\item It has already been shown using integration that the moment of inertia of one rod about its end is $\frac{1}{3}ml^2$, and so to find the total moment of inertia of the windmill you just add the contributions from each of the rods, giving $\frac{4}{3}ml^2$. 
\item Figure \ref{fig:AngularMotion_windmill} shows the initial set up:
\begin{figure}[h] 
\centering
\includegraphics[width=6cm]{AngularMotion_windmill}
\caption{}
\label{fig:AngularMotion_windmill}
\end{figure}
\\
and Figure \ref{fig:AngularMotion_windmillcollone} shows what happens immediately before,  and Figure \ref{fig:AngularMotion_windmillcolltwo} immdiately after the collision:
\begin{figure}[h]
\centering
\begin{subfigure}{0.5\textwidth}
\centering
\includegraphics[width=0.5\linewidth]{AngularMotion_windmillcollone}
\caption{}
\label{fig:AngularMotion_windmillcollone}
\end{subfigure}%
\begin{subfigure}{0.5\textwidth}
\centering
\includegraphics[width=0.5\linewidth]{AngularMotion_windmillcolltwo}
\caption{}
\label{fig:AngularMotion_windmillcolltwo}
\end{subfigure}
\caption{}
\end{figure}
\\
First conserve angular momentum:
\begin{align*}
lmu&=I\omega-lmv \\
\Rightarrow lm(u+v)&=\frac{4}{3}ml^2\omega \\
\Rightarrow u+v&=\frac{4}{3}l\omega
\end{align*}
and then energy:
\begin{align*}
\frac{1}{2}mu^2&=\frac{1}{2}mv^2+\frac{1}{2}I\omega^2 \\
\Rightarrow u^2&=v^2+\frac{4}{3}l^2\omega^2
\end{align*}
Now we just need to solve the two simultaneous equations. There are many ways to do this; I will substitute $\omega=\frac{3(u+v)}{4l}$ into the energy equation to obtain
\begin{align*}
u^2&=v^2+\frac{4}{3}l^2\cdot\left(\frac{3(u+v)}{4l}\right)^2 \\
\Rightarrow u^2&=v^2+\frac{3}{4}(u+v)^2 \\
\end{align*}
which rearranges to
\begin{align*} 7v^2+6uv-u^2&=0 \end{align*}
solve this using the quadratic formula:
\begin{align*}
\frac{v}{u}&=\frac{-6\pm\sqrt{36+28}}{14} \\
\Rightarrow \frac{v}{u}&=\frac{6}{14}\pm\frac{8}{14} \\
\Rightarrow v&=-u \textrm{ or } v=u/7
\end{align*}
Since we initially defined $v$ as being upwards, in the opposite direction to $u$, the first solution corresponds to the particle continuing on at the same speed; this obviously conserves momentum and energy, but would mean that $\omega=0$ and the mass has passed straight through the windmill, so it is not the solution we are looking for. Using the other solution we can find what $\omega$ is:
\begin{align*}
u+v&=\frac{4}{3}\omega \\
\Rightarrow u+u/7&=\frac{4}{3}\omega \\
\Rightarrow \omega&=\frac{6}{7}u
\end{align*} 
we can also substitute a value in for $u$ since we know what height the mass started at (notice how leaving it as $u$ until this point makes the algebra much simpler)
\begin{align*}
mgh&=\frac{1}{2}mu^2 \\
\Rightarrow u&=\sqrt{2gh}
\end{align*}
and so our final expression for $\omega$ is
\begin{align*}
\omega=\frac{6\sqrt{2gh}}{7}
\end{align*} 
\item It is very simple to calculate the height the ball rebounds to; we just conseve energy again
\begin{align*}
\frac{1}{2}mv^2&=mgy \\
\Rightarrow y&=\frac{1}{2g}v^2 \\
&=\frac{1}{2g}\cdot\left(\frac{\sqrt{2gh}}{7}\right)^2 \\
\Rightarrow y&=\frac{h}{49}
\end{align*}
\end{enumerate}
}
\end{hint}