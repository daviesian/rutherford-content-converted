%% ID: horizontal_ball
%% TITLE: A Ball Thrown off a Cliff 
%% TYPE: question
%% QUESTIONTYPE: numeric
%% CONCEPTS: forces, eq_of_motions_diff, newtonii, vectors1
%% VIDEOS: 
%% LEVEL: 2
%% TOPIC: mechanics/kinematics
%% ORDER: 5

\begin{problem}[Tripos2007_kinematics] %video?
{\exposition{A ball is thrown horizontally with a speed of \quantity{30}{m\,s\sup{-1}} from a height of \quantity{70}{m}.} \question{How far has the ball travelled horizontally when it first hits the ground?}
\vspace{-0.3cm}
}
{\textit{Cambridge University Tripos 2007}}
{Draw a diagram to start with, such as Figure \ref{fig:Kinematics_horiproj}
\begin{figure}[h]
\centering
\includegraphics[width=5cm]{Kinematics_horiproj}
\caption{}
\label{fig:Kinematics_horiproj}
\end{figure}
\\
Now use the SUVAT equations. In the vertical direction, the acceleration is $g$ downwards so
\begin{align*}
y=h-\frac{1}{2}gt^2 \\
\Rightarrow t=\sqrt{\frac{2h}{g}}
\end{align*}
since we want the time taken until it hits the ground, where $y$ is zero. The distance travelled in the horizontal direction is 
\begin{align*}
x&=vt \\
\Rightarrow x&=v\sqrt{\frac{2h}{g}}=113\textrm{ m}
\end{align*}
\answer{The ball has travelled \quantity{113}{m} horizontally when it first hits the ground}
}
\end{problem}