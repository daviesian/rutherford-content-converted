%% ID: the_lift
%% TITLE: The Lift
%% TYPE: question
%% QUESTIONTYPE: symbolic
%% CONCEPTS: momentum, energy, eq_of_motion_diff, newtonii
%% LEVEL: 2
%% TOPIC: mechanics/dynamics
%% ORDER: 9


\begin{problem}[HE+_Lift]% Needs N2L, a little SUVAT and Work.
{\exposition{A lift, of  mass \vari{m}, is travelling downwards at a speed \vari{u}. It is brought to rest by a constant acceleration over a distance \vari{h}.} 
\begin{enumerate} 
	\item \answer[a]{What is the tension, \vari{T}, in the lift cable when the lift is stopping?}
	\item \answer[b]{What is the work done by the tension whilst stopping the lift?}
\item \exposition{A lift, of mass \vari{m}, is travelling upwards at a speed \vari{u}. It is brought to rest by a constant acceleration over a distance \vari{h}.} 
\begin{enumerate}
	\item \answer[c]{What is the tension, \vari{T}, in the lift cable when the lift is stopping?}
	\item \answer[d]{What is the work done by the tension whilst stopping the lift?}
\end{enumerate}
NB: For the website, we are planning to have these two questions alternate at random. When a student clicks on the question he could receive either part a) or part b).
\end{enumerate}
}
{\stress{Used with permission from HE+.}}
{\begin{enumerate}
\item
\begin{enumerate}
	\item This part has two steps; find the acceleration, then find the force. We can use the SUVAT equations to find the acceleration; the equation \valuedef{v^{2}}{u^{2} + 2as}{} is the one required here. We have \valuedef{v}{0}{m\,s\sup{-1}}, \valuedef{u}{u}{} and \valuedef{s}{h}{}:
\begin{eqnarray*} 
a &= \frac{v^{2} - u^{2}}{2s} \\ 
&= -\frac{u^{2}}{2h} 
\end{eqnarray*}
And then by considering the total downward force we can find \vari{T}:
\begin{eqnarray*} 
F &= (mg - T) \\ 
&= ma = -\frac{mu^{2}}{2h} 
\end{eqnarray*}
or, rearranging 
\begin{eqnarray*} 
\answer[a]{T &= mg - ma \\ 
&= mg + \frac{mu^{2}}{2h}}
\end{eqnarray*}
	\item The work done by the tension is the force multiplied by the distance travelled in the direction the force acts: \vari{W} = \valuedef{\vtr{F}\cdot\vtr{d}}{-Th}{} (as the tension acts in the opposite direction to the distance).
	\begin{eqnarray*} 
	\text{Work} &= -Th \\
	&= -\left(mg + \frac{mu^{2}}{2h}\right)h \\ 
	\answer[b]{\text{Work}&= - mgh - \frac{1}{2}mu^{2}}
	\end{eqnarray*}
and it should be clear this equals the total change in energy of the lift.
\end{enumerate}
\item 
\begin{enumerate} 
\item This part has two steps; find the acceleration, then find the force. We can use the SUVAT equations to find the acceleration; the equation \valuedef{v^{2}}{u^{2} + 2as}{} is the one required here. We have \valuedef{v}{0}{m\,s\sup{-1}}, \valuedef{u}{u}{} and \valuedef{s}{h}{}:
\begin{eqnarray*} 
a &= \frac{v^{2} - u^{2}}{2s} \\ 
&= -\frac{u^{2}}{2h} 
\end{eqnarray*}
And then by considering the total upward force we can find \vari{T}:
\begin{eqnarray*} 
F &= (T - mg) \\ 
&= ma = -\frac{mu^{2}}{2h} 
\end{eqnarray*}
or, rearranging:
\begin{eqnarray*} 
\answer[c]{T &= mg + ma \\ 
&= mg - \frac{mu^{2}}{2h}} 
\end{eqnarray*}
	\item The work done by the tension is the force multiplied by the distance travelled in the direction the force acts: \vari{W} = \valuedef{\vtr{F}\cdot\vtr{d}}{Th}{}.
	\begin{eqnarray*} 
	\text{Work} &= Th \\
	&= \left(mg - \frac{mu^{2}}{2h}\right)h \\ 
	\answer[d]{\text{Work} &= mgh - \frac{1}{2}mu^{2}} 
	\end{eqnarray*}
and it should be clear this equals the total change in energy of the lift. 
\end{enumerate}
\end{enumerate}}
\end{problem}