%% ID: orbiting_moon
%% TITLE: Orbiting the Moon
%% TYPE: question
%% QUESTIONTYPE: numeric
%% CONCEPTS: angular_circular, energy
%% VIDEOS: IntA1988PIIQ3l.mp4
%% LEVEL: 4
%% TOPIC: mechanics/circular
%% ORDER: 5
\begin{problem}[IntA1988PIIQ3l]
{Find the speed of a satellite in a circular orbit around the Moon, very near the Moon's surface. What is the kinetic energy per unit mass of the satellite?
The gravitational force $F_A$ between the satellite and the Moon is in the inward radial direction and its magnitude is given by the equation 
\begin{equation*}
F_A=GMm/R^2
\end{equation*}
where $G$ is the gravitational constant; $M$ and $m$ are the masses of the Moon and the satellite respectively; and $R$ is the radius of the orbit. Give your answer to two decimal places.

\emph{Hint:} The radius of the Moon is $1.74\times 10^6\textrm{ m}$, the mass of the Moon is $7.35\times 10^{22}\textrm{ kg}$ and Newton's Gravitational Constant, $G$, is $6.67 \times 10^{-11}$ N m$^{2}$ kg$^{-2}$.} 
{\textit{Used with permission from UCLES, A Level Physics, November 1988, Paper 2, Question 3.}}
{In order to perform circular motion the magnitude of the gravitational force $F_g$ in the inward radial direction, 
\begin{equation*} F_g=GMm/R^2 \end{equation*}
must be equal to the magnitude of the centripetal force $F_c$ given by 
\begin{equation*} F_c=mv^2/R \end{equation*}
where $M$ is the mass of the moon, $m$ is the mass of the satellite, $R$ is the radius of the orbit (equal to the radius of the moon) and $v$ is the speed of the satellite. Setting the two expressions equal to each other gives
\begin{align*} GMm/R^2=mv^2/R \\ \Rightarrow v^2=GM/R \end{align*}
Therefore the speed is 
\begin{equation*} v=\sqrt{GM/R}=1.68\textrm{ km s}^{-1} \end{equation*} 
%and the kinetic energy per unit mass is \begin{equation*} \frac{1}{2}v^2=GM/2R=1.41\times 10^6\textrm{ J kg}^{-1} \end{equation*}
}
\end{problem}