%% ID: firing_the_rockets
%% TITLE: Firing the Rockets
%% TYPE: question
%% QUESTIONTYPE: scq
%% CONCEPTS: newtoni, newtonii, angular_circular, vectors, resolving_vectors
%% VIDEOS: A1988PIQ7l.mp4
%% LEVEL: 4
%% TOPIC: mechanics/circular
%% ORDER: 2
\begin{problem}[A1988PIQ7l] 
{An artificial satellite travels in a circular orbit about the Earth. Its rocket engine is then fired and produces a force on the satellite exactly equal and opposite to that exerted by the Earth's gravitational field. The satellite would then start to move:
\begin{enumerate}
	\item along a spiral path towards the Earth's surface.
	\item along the line joining it to the centre of the Earth (i.e. radially).
	\item along a tangent to the orbit. \answer
	\item in a circular orbit with a longer period.
	\item in a circular orbit with a shorter period.
\end{enumerate}}
{\textit{Used with permission from UCLES, A Level Physics, June 1988, Paper 1, Question 7.}}
{The correct answer is (c). The only force acting on the satellite initially is the gravitational force from Earth. If this force is cancelled out by an equal and opposite force from the engine, then there is no net force acting on the satellite and Newton's First Law then says that it continues to move with the velocity it had at the moment the net force became zero. This velocity was tangential, and so the satellite would begin to move tangentially.}
\end{problem}
