%% ID: acc_stone
%% TITLE: Acceleration of a Thrown Stone
%% TYPE: question
%% QUESTIONTYPE: scq
%% CONCEPTS: forces, newtonii
%% VIDEOS: 
%% LEVEL: 2
%% TOPIC: mechanics/kinematics
%% ORDER: 1

\begin{problem}[IntA1984PIIQ2l] 
{\exposition{In the absence of air resistance, a stone is thrown from $P$ and follows a parabolic path. The highest point reached is $T$.} \question{The vertical component of acceleration of the stone...}
\begin{enumerate}
	\item \choice[a]{is zero at $T$.}
	\item \choice[b]{is greatest at $T$.}
	\item \choice[c]{is greatest at $P$.}
	\item \choice[d]{is the same at $P$ as at $T$.}\correct
	\item \choice[e]{decreases at a constant rate.}
\end{enumerate}
}
{\textit{Used with permission from UCLES, A Level Physics, November 1984, Paper 2, Question 2.}}
{\answer{The correct answer is (d). The acceleration due to gravity, $g$, is the same at all points during the motion.}
}
\end{problem}