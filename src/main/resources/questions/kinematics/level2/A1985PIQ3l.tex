\begin{problem}[A1985PIQ3l] 
{The velocity of a car which is decelerating uniformly changes from 30 ms$^{-1}$ to 15 ms$^{-1}$ in 75 m. After what further distance will it come to rest?
\begin{enumerate}
	\item 25 m
	\item 37.5 m
	\item 50 m
	\item 75 m
	\item100 m
\end{enumerate}
}
{\textit{Used with permission from UCLES, A Level Physics, June 1985, Paper 1, Question 3.}}
{The answer is (a). The equation $v^{2} = u^{2} + 2as$ can be used to find the value of the car's deceleration. Rearranging gives $a = \frac{v^{2} - u^{2}}{2s} = \frac{225 - 900}{150} = -4.5$ ms$^{-2}$. We can then use the equation again to find the distance travelled until the car comes to rest, where $u = 15$ and $v = 0$. Rearrange the equation again to give $s = \frac{v^{2} - u^{2}}{2a} = \frac{-225}{-9} = 25$ m.
}
\end{problem}