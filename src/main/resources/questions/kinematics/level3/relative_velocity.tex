%% ID: relative_velocity
%% TITLE: Relative Velocity
%% TYPE: question
%% QUESTIONTYPE: scq
%% CONCEPTS: vectors1
%% VIDEOS: 
%% LEVEL: 3
%% TOPIC: mechanics/kinematics
%% ORDER: 3

\begin{problem}[Relative_Velocity_Multi] 
{\exposition{Spheres A and B are moving towards each other on the same line with speeds \quantity{4}{ ms\sup{-1}} and  \quantity{6}{ ms\sup{-1}} respectively. See Figure \ref{fig:Kinematics_Relative_Velocity_Multi}.} \question{What is the velocity of A relative to B? Take A's direction of motion to be positive.}

\begin{figure}[h]
	\centering
	\includegraphics[width=0.37\textwidth]{Kinematics_Relative_Velocity_Multi}
	\caption{}
	\label{fig:Kinematics_Relative_Velocity_Multi}
\end{figure}

\begin{enumerate}
	\item \choice[a]{\quantity{-2}{ ms\sup{-1}}}
	\item  \choice[b]{\quantity{-10}{ ms\sup{-1}}}
	\item  \choice[c]{\quantity{2}{ ms\sup{-1}}}
	\item  \choice[d]{\quantity{-4}{ ms\sup{-1}}}
	\item  \choice[e]{\quantity{10}{ ms\sup{-1}}}\correct
\end{enumerate}
}
{\textit{Written by JZ for The Rutherford Schools Physics Project.}}
{\answer{The answer is (e).} Since they are moving towards each other, each of their velocities relative to the other is $6+4 =10 \textrm{ ms}^{-1}$. Relative to B, A is moving to in the positive direction. Hence the answer is  \quantity{+10}{ ms\sup{-1}}. Be careful with directions; B's velocity relative to A is  \quantity{-10}{ ms\sup{-1}}.
}
\end{problem}